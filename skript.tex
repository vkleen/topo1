\documentclass[11pt,DIV=11,paper=a4]{scrartcl}
\usepackage[colorlinks=true,pdfencoding=auto,psdextra]{hyperref}

\usepackage[utf8]{inputenc}
\usepackage{xspace, xifthen, enumitem}
\usepackage{amssymb, amsmath, amsthm, thmtools, nth}
\usepackage{graphicx}
\usepackage{etoolbox}
\usepackage{empheq}
\usepackage{comment}
\usepackage{needspace}

\usepackage{tikz}
\usetikzlibrary{arrows,decorations.markings,chains,calc,matrix,cd}
\tikzset{>=cm to}

\usetikzlibrary{external}
% \tikzexternalize[prefix=tikz/,mode=list and make]

%\url{http://tex.stackexchange.com/q/171931/86}
\usepackage{environ}
\makeatletter
\def\tikzcd@[#1]{%
  \begin{tikzpicture}[/tikz/commutative diagrams/.cd,every diagram,#1]%
  \ifx\arrow\tikzcd@arrow%
    \pgfutil@packageerror{tikz-cd}{Diagrams cannot be nested}{}%
  \fi%
  \let\arrow\tikzcd@arrow%
  \let\ar\tikzcd@arrow%
  \def\rar{\tikzcd@xar{r}}%
  \def\lar{\tikzcd@xar{l}}%
  \def\dar{\tikzcd@xar{d}}%
  \def\uar{\tikzcd@xar{u}}%
  \def\urar{\tikzcd@xar{ur}}%
  \def\ular{\tikzcd@xar{ul}}%
  \def\drar{\tikzcd@xar{dr}}%
  \def\dlar{\tikzcd@xar{dl}}%
  \global\let\tikzcd@savedpaths\pgfutil@empty%
  \matrix[%
    /tikz/matrix of \iftikzcd@mathmode math \fi nodes,%
    /tikz/every cell/.append code={\tikzcdset{every cell}},%
    /tikz/commutative diagrams/.cd,every matrix]%
  \bgroup}

\def\endtikzcd{%
  \pgfmatrixendrow\egroup%
  \pgfextra{\global\let\tikzcdmatrixname\tikzlastnode};%
  \tikzcdset{\the\pgfmatrixcurrentrow-row diagram/.try}%
  \begingroup%
    \pgfkeys{% `quotes' library support
      /handlers/first char syntax/the character "/.initial=\tikzcd@forward@quotes,%
      /tikz/edge quotes mean={%
        edge node={node [execute at begin node=\iftikzcd@mathmode$\fi,%$
                         execute at end node=\iftikzcd@mathmode$\fi,%$
                         /tikz/commutative diagrams/.cd,every label,##2]{##1}}}}%
    \let\tikzcd@errmessage\errmessage% improve error messages
    \def\errmessage##1{\tikzcd@errmessage{##1^^J...^^Jl.\tikzcd@lineno\space%
        I think the culprit is a tikzcd arrow in cell \tikzcd@currentrow-\tikzcd@currentcolumn}}%
    \tikzcd@before@paths@hook%
    \tikzcd@savedpaths%
  \endgroup%
  \end{tikzpicture}%
  \ifnum0=`{}\fi}

\NewEnviron{mytikzcd}[1][]{%
  \def\@temp{\tikzcd@[#1]\BODY}%
  \expandafter\@temp\endtikzcd
}
\makeatother

\def\temp{&} \catcode`&=\active \let&=\temp

\usepackage[lining]{libertine}
\usepackage[T1]{fontenc}
\usepackage{textcomp}
\usepackage[varqu,varl]{inconsolata}
%\usepackage[italic, basic, eulergreek, defaultmathsizes]{mathastext}
\usepackage[libertine]{newtxmath}
%\usepackage{libertinust1math}
\usepackage{bm}
\usepackage{mathtools}
\mathtoolsset{mathic}

\usepackage[cal=boondoxo]{mathalfa}
\usepackage{mathrsfs}
\newcommand{\mathcalls}[1]{{\textls*[-150]{\usefont{U}{BOONDOX-calo}{m}{n} #1}}}

\usepackage[all]{hypcap}

\usepackage{csquotes}
\usepackage[german]{babel}
\usetikzlibrary{babel}
\usepackage[nodayofweek]{datetime}

\usepackage[protrusion=true]{microtype}

% \usepackage[bibencoding=utf8,style=alphabetic,citestyle=alphabetic,backref=true,hyperref=true,giveninits=true,doi=true]{biblatex}
% \addbibresource{../all.bib}
% \renewcommand*{\bibfont}{\normalfont\footnotesize}
% \renewbibmacro{in:}{%
%   \ifentrytype{article}{}{\printtext{\bibstring{in}\intitlepunct}}}
% \renewrobustcmd*{\bibinitdelim}{\,}
% \AtEveryBibitem{%
%   \clearfield{pagetotal}%
% }

\usepackage{pdfpages}

\declaretheoremstyle[spaceabove=\topsep,spacebelow=\topsep,headfont=\normalfont\scshape,notefont=\normalfont\mdseries,notebraces={(}{)},bodyfont=\normalfont,postheadspace=5pt plus 1pt minus 1pt]{scdef}
\declaretheoremstyle[spaceabove=\topsep,spacebelow=\topsep,headfont=\normalfont\scshape,notefont=\normalfont\mdseries,notebraces={(}{)},bodyfont=\itshape,postheadspace=5pt plus 1pt minus 1pt]{scthm}

\declaretheorem[style=scdef,numberwithin=section,   name=Definition,refname={Definition,Definitions},Refname={Definition,Definitions}]{definition}
\declaretheorem[style=scdef,sharenumber=definition, name=Bemerkung,refname={Bemerkung,Bemerkungen},Refname={Bemerkung,Bemerkungen}]{remark}
\declaretheorem[style=scdef,sharenumber=definition, name=Beispiel,refname={Beispiel,Beispiele},Refname={Beispiel,Beispiele}]{example}

\newcounter{Sheet}
\declaretheorem[style=scdef,numberwithin=Sheet,name=Aufgabe,refname={Aufgabe,Aufgaben},Refname={Aufgabe,Aufgaben}]{exercise}

\declaretheorem[style=scthm,sharenumber=definition, name=Satz,refname={Satz,Sätze},Refname={Satz,Sätze}]{theorem}
\declaretheorem[style=scthm,numbered=no, name=Satz,refname={Satz,Sätze},Refname={Satz,Sätze}]{theorem*}
\declaretheorem[style=scthm,sharenumber=definition, name=Lemma,refname={Lemma,Lemmata},Refname={Lemma,Lemmata}]{lemma}
\declaretheorem[style=scthm,sharenumber=definition, name=Korollar,refname={Korollar,Korollare},Refname={Korollar,Korollare}]{corollary}

\undef\Re
\undef\Im

\newcommand*{\normal}{\lhd}
\newcommand*{\isom}{\cong}
\newcommand*{\homot}{\sim}
\newcommand*{\wequiv}{\simeq}

\makeatletter
\let\@oldsubset=\subset
\def\@subsethelper#1#2{\mathrel{\raisebox{.5pt}{$#1\@oldsubset$}}\xspace}
\DeclareRobustCommand*{\subset}{\mathpalette\@subsethelper\relax}

\let\@oldotimes=\otimes
\def\@otimeshelper#1#2{\mathrel{\raisebox{.5pt}{$#1\@oldotimes$}}\xspace}
\DeclareRobustCommand*{\otimes}{\mathpalette\@otimeshelper\relax}

\let\@oldsetminus=\setminus
\undef\setminus
\DeclareRobustCommand*{\setminus}{\mathpalette\smallsetminus\relax}
\makeatother

\tikzset{/tikz/commutative diagrams/arrows={thin}}

\newcommand*{\ddh}{d.\,h.}
\newcommand*{\ie}{i.\,e.}
\newcommand*{\eg}{e.\,g.}
\newcommand*{\Ie}{I.\,e.}
\newcommand*{\Eg}{E.\,g.}

\undef\lrcorner
\newcommand{\lrcorner}{\mathord{\vrule height 0.1ex depth 0pt width 1ex\vrule height 1.3ex depth 0pt width
0.1ex}}

\def\<#1>{\left\langle #1 \right\rangle}

\undef\AA
\undef\SS
\renewcommand*{\do}[1]{\expandafter\def\csname#1#1\endcsname{\ensuremath{\mathbb{#1}}\xspace}}
\docsvlist{A,B,C,D,E,F,G,H,I,J,K,L,M,N,O,P,Q,R,S,T,U,V,W,X,Y,Z}

\setlist[enumerate]{label={\normalfont \rmfamily(\roman*)}, nosep}
\setlist[itemize]{nosep}
\setlist[description]{font={\normalfont}}

\overfullrule=1mm

\usepackage{todonotes}
\newcommand{\aravind}[1]{\todo[color=red!40]{#1}} %notes by Aravind
\newcommand{\marc}[1]{\todo[color=blue!40]{#1}} %notes by Marc
\newcommand{\matthias}[1]{\todo[color=cyan!40]{#1}} %notes by Matthias
\newcommand{\brad}[1]{\todo[color=magenta!40]{#1}} %notes by Brad
\newcommand{\viktor}[1]{\todo[color=yellow!40]{#1}} %notes by Viktor

\ifdefined\RedeclareSectionCommand
\RedeclareSectionCommand[counterwithin=section,font={\normalsize\normalfont\scshape}]{paragraph}
\fi

\renewcommand{\sectionautorefname}{Section}
\renewcommand{\chapterautorefname}{Chapter}

\renewcommand*\titlepagestyle{empty}

\undef\lim
\DeclareMathOperator*{\lim}{\textnormal{lim}}
\DeclareMathOperator*{\colim}{\textnormal{colim}}
\DeclareMathOperator*{\hocolim}{\textnormal{hocolim}}
\DeclareMathOperator*{\holim}{\textnormal{holim}}
\DeclareMathOperator*{\hofib}{\textnormal{hofib}}

\DeclareMathOperator{\sk}{\textnormal{sk}}
\DeclareMathOperator{\cosk}{\textnormal{cosk}}

\undef\hom
\DeclareMathOperator{\hom}{\textnormal{Hom}}
\DeclareMathOperator{\End}{\textnormal{End}}
\DeclareMathOperator{\map}{\textnormal{Map}}
\DeclareMathOperator{\Map}{\textnormal{\bfseries Map}}
\DeclareMathOperator{\aut}{\textnormal{Aut}}
\DeclareMathOperator{\Hom}{\textnormal{\bfseries Hom}}
\DeclareMathOperator{\Aut}{\textnormal{\bfseries Aut}}
\DeclareMathOperator{\RHom}{\mathbb{R}\textnormal{\bfseries Hom}}

\DeclareMathOperator{\spshv}{\textnormal{\textsf{sPShv}}}
\DeclareMathOperator{\sset}{\textnormal{\textsf{sSet}}}
\DeclareMathOperator{\set}{\textnormal{\textsf{Set}}}
\DeclareMathOperator{\grp}{\textnormal{\textsf{Grp}}}
\DeclareMathOperator{\abgrp}{\textnormal{\textsf{AbGrp}}}
\DeclareMathOperator{\simplex}{\mathbf{\Delta}}
\DeclareMathOperator{\wbar}{\overline W}
\DeclareMathOperator{\sgrp}{\textnormal{\textsf{sGrp}}}
\DeclareMathOperator{\sgpd}{\textnormal{\textsf{sGpd}}}

\DeclareMathOperator{\Spec}{\textnormal{Spec}}

\newcommand{\EGL}{\mathord{\textnormal{EGL}}}
\newcommand{\BGL}{\mathord{\textnormal{BGL}}}
\newcommand{\MGL}{\mathord{\textnormal{MGL}}}
\newcommand{\BU}{\mathord{\textnormal{BU}}}
\newcommand{\BO}{\mathord{\textnormal{BO}}}
\newcommand{\Unitary}{\mathord{\textnormal{U}}}

\newcommand{\GL}{\mathord{\textnormal{GL}}}
\newcommand{\Gr}[2][]{\ifthenelse{\isempty{#1}}{%
  \mathord{\textnormal{Gr}}_{#2}}{%
  \operatorname{\textnormal{Gr}}_{#2}(#1)}}

\newcommand{\smashp}{\wedge}

\newcommand{\Tsusp}[1][]{\Sigma_{\PP^1}\ifthenelse{\isempty{#1}}{^\infty}{^{#1}}}
\newcommand{\transfer}[1][]{\operatorname{tr}\ifthenelse{\isempty{#1}}{}{_{#1}}}
\newcommand{\Transfer}[1][]{\operatorname{Tr}\ifthenelse{\isempty{#1}}{}{_{#1}}}
\newcommand{\Tr}[1]{\operatorname{Tr}(#1)}
\newcommand{\proj}[1][]{\operatorname{\textnormal{proj}}\ifthenelse{\isempty{#1}}{}{_{#1}}}
\newcommand{\one}[1]{\mathbf{1}_{#1}}

\newcommand{\red}{\textnormal{red}}
\newcommand{\Nis}{\textnormal{Nis}}
\newcommand{\mot}{\textnormal{mot}}

\newcommand*{\Aone}[1][1]{\AA\mkern-1mu^{#1}}

\makeatletter
\newcommand{\@SH}{\operatorname{\cal{S\mkern-1mu H}}}
\newcommand{\@H}{\operatorname{\cal{H}}}
\newcommand{\@Spc}{\operatorname{\cal{S\mkern-2mup\mkern-3muc}}}
\newcommand{\@Fun}{\operatorname{\cal{F\mkern-2muu\mkern-4mun}}}
\newcommand{\SH}[1][]{\ifthenelse{\isempty{#1}}{\@SH}{\@SH\mkern-1mu(#1)}}
\newcommand{\Ho}[1][]{\ifthenelse{\isempty{#1}}{\@H}{\@H\mkern-1mu(#1)}}
\newcommand{\Spc}[1]{\ifthenelse{\isempty{#1}}{\@Spc}{\@Spc\mkern-1mu(#1)}}
\newcommand{\SpcNis}[1]{\ifthenelse{\isempty{#1}}{\@Spc_\Nis}{\@Spc_\Nis(#1)}}
\newcommand{\SpcA}[1]{\ifthenelse{\isempty{#1}}{\@Spc\mkern-5mu_{\Aone}}{\@Spc\mkern-5mu_{\Aone}(#1)}}
\newcommand{\Spt}[1]{\ifthenelse{\isempty{#1}}{\@Spt}{\@Spt\mkern-1mu(#1)}}
\newcommand{\Fun}{\@Fun\mkern-1mu}

\newcommand{\@Sch}{\operatorname{\textnormal{Sch}}}
\newcommand{\Sch}[1][]{\ifthenelse{\isempty{#1}}{\@Sch}{\@Sch_{/#1}}}
\DeclareMathOperator{\Ind}{\textnormal{Ind}}
\makeatother

\DeclareMathOperator{\LocNis}{\textnormal{L}_{\Nis}}
\DeclareMathOperator{\LocA}{\textnormal{L}_{\Aone}}
\DeclareMathOperator{\LocMot}{\textnormal{L}_{\mot}}

\newcommand{\Catinfty}{\cal{C\mkern-3mua\mkern-4mut}_\infty}

\newcommand{\Sm}[1]{\operatorname{\textnormal{Sm}}_{#1}}
\newcommand{\sm}{\textnormal{sm}}
\newcommand{\Aff}[1]{\operatorname{\textnormal{Aff}}_{#1}}
\newcommand{\PrL}{\cal{P\mkern-4mu r}^{\textnormal{L}}}
\newcommand{\PrR}{\cal{P\mkern-4mu r}^{\textnormal{R}}}

\newcommand{\Cl}{\textnormal{cl}}
\newcommand{\Op}{\textnormal{op}}

\DeclareMathOperator{\Shv}{\textnormal{Shv}}

\DeclareMathOperator{\fib}{\textnormal{fib}}
\DeclareMathOperator{\cof}{\textnormal{cof}}

\DeclareMathOperator{\id}{\textnormal{id}}

\DeclareMathOperator{\unit}{\textnormal{unit}}
\DeclareMathOperator{\counit}{\textnormal{counit}}
\DeclareMathOperator{\Ex}{\textnormal{Ex}}

\DeclareMathOperator{\incl}{\textnormal{incl}}

\newcommand{\trunc}[1]{\tau_{#1}}
\DeclareMathOperator{\disc}{\textnormal{Disc}}
\newcommand{\Cof}[1]{{#1}^{cf}}

\newcommand{\Deloop}[2][]{\ifthenelse{\isempty{#1}}{\textnormal{\bfseries B}#2}{\textnormal{\bfseries B}^{#1}#2}}
\newcommand{\deloop}[2][]{\ifthenelse{\isempty{#1}}{\textnormal{B}#2}{\textnormal{B}^{#1}#2}}
\newcommand{\actgrp}[2]{#1/\!/#2}

\DeclareMathOperator{\ev}{\textnormal{ev}}

\DeclareMathOperator{\EM}{\textnormal{\textsf{EM}}}
\DeclareMathOperator{\hoEM}{\textnormal{\textsf{hEM}}}

\newcommand{\kanloop}[1]{\ifthenelse{\isempty{#1}}{\operatorname{\GG}}{\GG#1}}

\newcommand{\Th}[1]{\ensuremath{#1}\textsuperscript{th}}
\newcommand{\Ths}[1]{\ensuremath{#1}\textsuperscript{st}}
\newcommand{\Thn}[1]{\ensuremath{#1}\textsuperscript{nd}}
\newcommand{\Thr}[1]{\ensuremath{#1}\textsuperscript{rd}}

\newcommand*{\infcat}{\(\infty\)–category\xspace}
\newcommand*{\infcats}{\(\infty\)–categories\xspace}

\newcommand{\ho}[1]{\textnormal{h}#1}

\DeclareMathOperator{\coev}{\textnormal{coev}}
\DeclareMathOperator{\switch}{\textnormal{switch}}
\newcommand*{\dual}[1]{{#1}^{\vee}}

\newcommand*{\Fin}{\textnormal{Fin}}

\let\setminus=\smallsetminus

\newcommand{\proofomitted}{\hfill\ensuremath{\blacksquare}}

\makeatletter
\newbox\@xrat
\renewcommand*{\xrightarrow}[2][-cm to]{%
  \setbox\@xrat=\hbox{\ensuremath{\scriptstyle #2}}
  \pgfmathsetlengthmacro{\@xratlen}{max(1.6em,\wd\@xrat+.6em)}
  \pgfmathsetlengthmacro{\@xratinnersep}{.5ex-\dp\@xrat}
  \mathrel{\tikz [#1,baseline=-.6ex]
    \draw (0,0) -- node[auto,inner sep=\@xratinnersep] {\box\@xrat} (\@xratlen,0) ;}}
\renewcommand*{\xleftarrow}[2][cm to-]{%
  \setbox\@xrat=\hbox{\ensuremath{\scriptstyle #2}}
  \pgfmathsetlengthmacro{\@xratlen}{max(1.6em,\wd\@xrat+.6em)}
  \pgfmathsetlengthmacro{\@xratinnersep}{.5ex-\dp\@xrat}
  \mathrel{\tikz [#1,baseline=-.6ex]
    \draw (0,0) -- node[auto,inner sep=\@xratinnersep] {\box\@xrat} (\@xratlen,0) ;}}
\newcommand*{\xrightarrowb}[2][-cm to]{%
  \setbox\@xrat=\hbox{\ensuremath{\scriptstyle #2}}%
  \pgfmathsetlengthmacro{\@xratlen}{max(1.6em,\wd\@xrat+.6em)}%
  \pgfmathsetlengthmacro{\@xratinnersep}{.5ex}%
  \mathrel{\tikz [#1,baseline=-.6ex]%
    \draw (0,0) -- node[auto,inner sep=\@xratinnersep] {\box\@xrat} (\@xratlen,0) ;}}
\newcommand*{\xleftarrowb}[2][cm to-]{%
  \setbox\@xrat=\hbox{\ensuremath{\scriptstyle #2}}
  \pgfmathsetlengthmacro{\@xratlen}{max(1.6em,\wd\@xrat+.6em)}
  \pgfmathsetlengthmacro{\@xratinnersep}{.5ex}
  \mathrel{\tikz [#1,baseline=-.6ex]
    \draw (0,0) -- node[auto,inner sep=\@xratinnersep] {\box\@xrat} (\@xratlen,0) ;}}

\pgfarrowsdeclare{my right hook}{my right hook}
{
\arrowsize=0.2pt
\advance\arrowsize by .5\pgflinewidth
\pgfarrowsleftextend{-.5\pgflinewidth}
\pgfarrowsrightextend{3.5\arrowsize+.5\pgflinewidth}
}
{
\arrowsize=0.2pt
\advance\arrowsize by .5\pgflinewidth
\pgfsetdash{}{0pt} % do not dash
\pgfsetroundjoin % fix join
\pgfsetroundcap % fix cap
\pgfpathmoveto{\pgfpoint{0\arrowsize}{-7\arrowsize}}
\pgfpatharc{-90}{90}{3.5\arrowsize}
\pgfusepathqstroke
}

\tikzset{%
  iso/.style={above,sloped,inner sep=0},
  iso'/.style={below,sloped,inner sep=0},
  to/.style={-cm to},
  from/.style={cm to-},
  onto/.style={-cm double to},
  into/.style={my right hook-cm to},
  mapsto/.style={|-cm to},
  clim/.style={decoration={markings,
                           mark=at position#1 with {\draw[-] (0,-3\pgflinewidth) -- (0,3\pgflinewidth);}},
               postaction=decorate},
  clim/.default=0.5,
  opim/.style={decoration={markings,
                           mark=at position#1 with {\draw[-] circle(3\pgflinewidth);}},
               postaction=decorate},
  opim/.default=0.5
}

\newcommand*\@tikzto[2]{\begin{tikzpicture}[baseline]%
      \draw[to,line width={#2\pgflinewidth},scale=#1](0,.55ex) -- (1.6em,.55ex);%
    \end{tikzpicture}}

\newcommand*\@tikzfrom[2]{\begin{tikzpicture}[baseline]%
      \draw[from,line width={#2\pgflinewidth},scale=#1](0,.55ex) -- (1.6em,.55ex);%
    \end{tikzpicture}}

\newcommand*\@tikzcto[2]{\mathrel{\begin{tikzpicture}[baseline]%
      \draw[to,line width={#2\pgflinewidth},scale=#1](0,.55ex) -- (0.8em,.55ex);%
    \end{tikzpicture}}}

\newcommand*\@tikzonto[2]{\mathrel{\begin{tikzpicture}[baseline]%
      \draw[onto,line width={#2\pgflinewidth},scale=#1](0,.55ex) -- (1.6em,.55ex);%
    \end{tikzpicture}}}

\newcommand*\@tikzinto[2]{\mathrel{\begin{tikzpicture}[baseline]%
      \draw[into,line width={#2\pgflinewidth},scale=#1](0,.55ex) -- (1.6em,.55ex);%
    \end{tikzpicture}}}

\newcommand*\@tikzclim[2]{\mathrel{\begin{tikzpicture}[baseline]%
      \draw[into,clim,line width={#2\pgflinewidth},scale=#1](0,.55ex) -- (1.6em,.55ex);%
    \end{tikzpicture}}}

\newcommand*\@tikzopim[2]{\mathrel{\begin{tikzpicture}[baseline]%
      \draw[into,opim,line width={#2\pgflinewidth},scale=#1](0,.55ex) -- (1.6em,.55ex);%
    \end{tikzpicture}}}

\newcommand*\@tikzmapsto[2]{\begin{tikzpicture}[baseline]%
      \draw[mapsto,line width={#2\pgflinewidth},scale=#1](0,.55ex) -- (1.6em,.55ex);%
    \end{tikzpicture}}

\newcommand*\@tikzcmapsto[2]{\begin{tikzpicture}[baseline]%
      \draw[mapsto,line width={#2\pgflinewidth},scale=#1](0,.55ex) -- (0.8em,.55ex);%
    \end{tikzpicture}}

\newcommand*\@tikziso[4]{\mathrel{\begin{tikzpicture}[baseline]%
      \draw[to,line width={#2\pgflinewidth},scale=#1](0,.55ex) -- node[iso,pos=0.47,inner sep=#4]{$#3\sim$} (1.6em,.55ex);%
    \end{tikzpicture}}}

\newcommand*\@tikzadjunction[2]{\begin{tikzpicture}[baseline]%
    \draw[to,line width={#2\pgflinewidth},scale=#1](0,1.7ex) -- node[pos=.5,below,inner sep=.3ex]{\rotatebox[origin=c]{90}{$\vdash$}} (1.6em,1.7ex);
    \draw[to,line width={#2\pgflinewidth},scale=#1](1.6em,-.3ex) -- (0,-.3ex);
  \end{tikzpicture}}

\newsavebox{\@todisplay}
\newsavebox{\@totext}
\newsavebox{\@toscript}
\newsavebox{\@toscriptscript}
\newsavebox{\@mapstodisplay}
\newsavebox{\@mapstotext}
\newsavebox{\@mapstoscript}
\newsavebox{\@mapstoscriptscript}
\newsavebox{\@cmapstodisplay}
\newsavebox{\@cmapstotext}
\newsavebox{\@cmapstoscript}
\newsavebox{\@cmapstoscriptscript}
\newsavebox{\@tikzadjunctiondisplay}
\newcommand*{\@saveboxes}{%
  \savebox{\@todisplay}{\@tikzto{1.0}{1}}%
  \savebox{\@totext}{\@tikzto{1.0}{1}}%
  \savebox{\@toscript}{\@tikzto{0.8}{0.9}}%
  \savebox{\@toscriptscript}{\@tikzto{0.8}{0.75}}%
  \savebox{\@mapstodisplay}{\@tikzmapsto{1.0}{1}}%
  \savebox{\@mapstotext}{\@tikzmapsto{1.0}{1}}%
  \savebox{\@mapstoscript}{\@tikzmapsto{0.8}{0.9}}%
  \savebox{\@mapstoscriptscript}{\@tikzmapsto{0.8}{0.75}}%
  \savebox{\@cmapstodisplay}{\@tikzcmapsto{1.0}{1}}%
  \savebox{\@cmapstotext}{\@tikzcmapsto{1.0}{1}}%
  \savebox{\@cmapstoscript}{\@tikzcmapsto{0.8}{0.9}}%
  \savebox{\@cmapstoscriptscript}{\@tikzcmapsto{0.8}{0.75}}%
  \savebox{\@tikzadjunctiondisplay}{\@tikzadjunction{1.0}{1}}%
}
\@saveboxes

\newcommand*\tikzto{\mathrel{\mathchoice{\usebox{\@todisplay}}%
  {\usebox{\@totext}}%
  {\usebox{\@toscript}}%
  {\usebox{\@toscriptscript}}}}

\newcommand*\tikzmapsto{\mathrel{\mathchoice{\usebox{\@mapstodisplay}}%
  {\usebox{\@mapstotext}}%
  {\usebox{\@mapstoscript}}%
  {\usebox{\@mapstoscriptscript}}}}

\newcommand*\tikzcmapsto{\mathrel{\mathchoice{\usebox{\@cmapstodisplay}}%
  {\usebox{\@cmapstotext}}%
  {\usebox{\@cmapstoscript}}%
  {\usebox{\@cmapstoscriptscript}}}}

\newcommand*{\tikzadjunction}{\mathrel{\usebox{\@tikzadjunctiondisplay}}}

\newcommand*\tikzfrom{\mathrel{\mathchoice{\@tikzfrom{1.0}{1}}{\@tikzfrom{1.0}{1}}{\@tikzfrom{0.8}{0.9}}{\@tikzfrom{0.6}{0.75}}}}
\newcommand*\tikzcto{\mathchoice{\@tikzcto{1.0}{1}}{\@tikzcto{1.0}{1}}{\@tikzcto{0.8}{0.9}}{\@tikzcto{0.6}{0.75}}}
\newcommand*\tikzonto{\mathchoice{\@tikzonto{1.0}{1}}{\@tikzonto{1.0}{1}}{\@tikzonto{0.8}{0.9}}{\@tikzonto{0.6}{0.75}}}
\newcommand*\tikzinto{\mathchoice{\@tikzinto{1.0}{1}}{\@tikzinto{1.0}{1}}{\@tikzinto{0.8}{0.9}}{\@tikzinto{0.6}{0.75}}}
\newcommand*\tikzclim{\mathchoice{\@tikzclim{1.0}{1}}{\@tikzclim{1.0}{1}}{\@tikzclim{0.8}{0.9}}{\@tikzclim{0.6}{0.75}}}
\newcommand*\tikzopim{\mathchoice{\@tikzopim{1.0}{1}}{\@tikzopim{1.0}{1}}{\@tikzopim{0.8}{0.9}}{\@tikzopim{0.6}{0.75}}}
\newcommand*\tikziso{\mathchoice{\@tikziso{1.0}{1}{\displaystyle}{0pt}}%
  {\@tikziso{1.0}{1}{\textstyle}{0pt}}%
  {\@tikziso{0.8}{0.9}{\scriptstyle}{0pt}}%
  {\@tikziso{0.67}{0.8}{\scriptscriptstyle}{0.15ex}}}

\let\@color\color
\renewcommand*{\color}[1]{\@color{#1}\@saveboxes}
\makeatother

\renewcommand*{\to}{\tikzto}
\newcommand*{\from}{\tikzfrom}
% \renewcommand*{\to}[1][]{\ifthenelse{\isempty{#1}}{\tikzto}{\xrightarrowb{#1}}}
% \newcommand*{\from}[1][]{\ifthenelse{\isempty{#1}}{\tikzfrom}{\xleftarrowb{#1}}}
\newcommand*{\cto}{\ensuremath{\tikzcto}}
\newcommand*{\cmapsto}{\ensuremath{\tikzcmapsto}}
\newcommand*{\into}{\tikzinto}
\newcommand*{\onto}{\tikzonto}
% \newcommand*{\into}[1][]{\ifthenelse{\isempty{#1}}{\tikzinto}{\xrightarrowb[into]{#1}}}
% \newcommand*{\onto}[1][]{\ifthenelse{\isempty{#1}}{\tikzonto}{\xrightarrowb[onto]{#1}}}
\newcommand*{\clim}{\tikzclim}
\newcommand*{\opim}{\tikzopim}

\newcommand*{\iso}{\tikziso}

\renewcommand*{\mapsto}{\tikzmapsto}

\newcommand*{\adjunction}[4]{#1 : #2 \tikzadjunction #3 : #4}



\title{Topologie}
\author{\normalsize Viktor Kleen \\[-1ex] \texttt{\footnotesize viktor.kleen@uni-due.de} %
   \and \normalsize Sabrina Pauli \\[-1ex] \texttt{\footnotesize sabrinp@math.uio.no}}
\date{}

\begin{document}
\maketitle
\section{Topologische Räume und stetige Funktionen}
Zuerst wollen wir Begriffe aus der Analysis wiederholen um später die Definition von topologischen Räumen zu motivien. Die ersten metrischen Räume, die man typischerweise antrifft, sind die Vektorräume $\RR^n$, die mit verschiedenen Normen ausgestattet werden können. Zum Beispiel definiert man die \emph{Supremumsnorm}
\[
 \|(x_1,\dots,x_n)\|_\infty = \max\{|x_1|,\dots,|x_n|\}
\]
oder für $1\leq p<\infty$ die $\ell^p$-Norm
\[
  \|(x_1,\dots,x_n)\|_p = (|x_1|^p + \dots + |x_n|^p)^{1/p}.
\]
\begin{definition}
Eine Abbildung $f\colon \RR^n\to\RR$ ist \emph{stetig} bezüglich einer Norm
$\|\_\|$ falls es zu jedem $x\in\RR^n$ und $\varepsilon > 0$ ein $\delta>0$
gibt, so dass $|f(x) - f(x')| < \varepsilon$ für alle $x'\in\RR^n$ mit $\|x-x'\|
< \delta$ gilt.
\end{definition}
\begin{theorem}
Je zwei Normen $\|_\|$ und $\|\_\|'$ auf $\RR^n$ sind \emph{äquivalent}, \ddh~es
gibt Konstanten $c,C>0$, so dass
\[
c\|x\| < \|x\|' < C\|x\|
\]
für alle $x\in \RR$.
\end{theorem}
\begin{corollary}\label{cor:continuity-equiv-norms}
Der Stetigkeitsbegriff für Funktionen $f\colon \RR^n\to\RR$ hängt nicht von der
gewählten Norm auf $\RR^n$ oder $\RR$ ab.
\end{corollary}
Dieses Korollar motiviert sofort die Frage, ob es einen Begriff von Stetigkeit
gibt, der von der Wahl einer Norm losgelöst ist? Die Antwort auf diese Frage ist
natürlich ja, aber wir werden dafür zuerst den Begriff einer \emph{Metrik} unter
suchen.

\begin{definition}
Eine \emph{Metrik} auf einer Menge $X$ ist eine Abbildung $d\colon X\times
X\to\RR$, so dass für $x,y,z\in X$ gilt:
\begin{enumerate}
\item $d(x,y)\geq 0$ und $d(x,y) = 0$ genau dann, wenn $x=y$.
\item $d(x,y) = d(y,x)$.
\item $d(x,z) \leq d(x,y) + d(y,z)$.
\end{enumerate}
Ein \emph{metrischer Raum} ist eine Menge $X$ zusammen mit einer Metrik $d$ auf
$X$.
\end{definition}

Zum Beispiel liefert jede Norm $\|\_\|$ auf einem Vektorraum $V$ eine Metrik
durch
\[
d(x,y) \coloneqq \|x-y\|.
\]
\begin{definition}
Seien $(X,d_X)$ und $(Y,d_Y)$ metrische Räume. Eine Abbildung $f\colon X\to Y$
heißt \emph{stetig}, falls es für jedes $x\in X$ und $\varepsilon>0$ ein
$\delta>0$ gibt, so dass $d_Y(f(x), f(x')) < \varepsilon$ für alle $x'\in X$ mit
$d_X(x,x')<\delta$ gilt.
\end{definition}
Diese Definition sieht erstmal nicht besonders hilfreich aus für unser Ziel
einen allgemeineren Begriff der Stetigkeit zu finden. Aber mit ihr können wir
beginnen eine Definition zu finden, die die Metrik nicht mehr explizit erwähnt.
\begin{definition}
Sei $(X,d)$ ein metrische Raum. Der \emph{offene Ball} um $x\in X$ mit Radius
$r>0$ ist
\[
B_r(x)\coloneqq \{y\in X : d(x,y) < r\}.
\]
Eine Teilmenge $U\subset X$ heißt \emph{offen}, falls für jedes $x\in U$ ein
$\varepsilon>0$ existiert, so dass $B_\varepsilon(x)\subset U$.
\end{definition}
Beispielsweise können wir $X=\RR$ mit der Metrik $d\colon X\times X\to\RR$,
$d(x,y) = |x-y|$ betrachten. Dann ist
\begin{itemize}
\item das offene Intervall $(a,b) = \{x\in\RR : a < x < b\}$ offen.
\item die Vereinigung zweier offener Intervalle $(a,b) \cup (c,d)$ offen.
\item das abgeschlossene Intervall $[a,b] = \{x\in\RR : a\leq x \leq b\}$
\emph{nicht} offen.
\end{itemize}
\begin{theorem}
Seien $(X,d_X)$ und $(Y,d_Y)$ metrische Räume und $f\colon X\to Y$ eine
Abbildung. Die folgenden Aussagen sind äquivalent:
\begin{enumerate}
\item $f$ ist stetig.
\item Für jede offene Teilmenge $U\subset Y$ ist $f^{-1}(U) = \{x\in X :
f(x)\in U\}$ offen.
\end{enumerate}
\end{theorem}
\begin{proof}
Sei zuerst $f$ stetig und $U\subset Y$ offen. Wir wollen zeigen, dass
$f^{-1}(U)$ offen ist, also dass für jedes $x\in f^{-1}(U)$ ein $\delta>0$
existiert, so dass $B_\delta(x)\subset f^{-1}(U)$. Aber $U$ ist offen, also
existiert ein $\varepsilon>0$, so dass $B_\varepsilon(f(x))\subset U$. Da $f$
stetig ist, gibt es tatsächlich ein $\delta > 0$, so dass $d(f(x),f(x')) <
\varepsilon$ für alle $x'\in X$ mit $d(x,x')<\delta$ gilt. Das bedeutet, dass $f(B_\delta(x))\subset B_\varepsilon(f(x))\subset U$,
oder anders gesagt, $B_\delta(x)\subset f^{-1}(B_\varepsilon(f(x)))\subset
f^{-1}(U)$.

Sei umgekehrt $x\in X$ und $\varepsilon>0$. Der offene Ball
$B_\varepsilon(f(x))$ ist offen (Übungsaufgabe!) und nach der Annahme an $f$ ist
damit auch $f^{-1}(B_\varepsilon(f(x)))$ offen und $x\in
f^{-1}(B_\varepsilon(f(x)))$. Also gibt es ein $\delta>0$, so dass
$B_\delta(x)\subset f^{-1}(B_\varepsilon(f(x)))$. Oder anders gesagt, für jedes
$x'\in B_\delta(x)$, \ddh~$d(x,x')<\delta$, ist $f(x')\in B_\varepsilon(f(x))$,
\ddh~$d(f(x),f(x')) < \varepsilon$.
\end{proof}
Mit diesem Satz haben wir einen vielversprechenden Kandidaten für einen
Stetigkeitsbegriff, denn Bedingung (ii) braucht nicht mehr explizit eine Metrik,
sondern nur noch den Begriff einer \emph{offenen Teilmenge}.
\subsection{Grundbegriffe}
\begin{definition}
Sei $X$ eine Menge. Eine \emph{Topologie} auf $X$ ist eine Menge $\cal T$ von
Teilmengen von $X$ mit
\begin{enumerate}
\item $\emptyset, X\in\cal T$
\item Für $U,V\in\cal T$ gilt $U\cap V\in\cal T$.
\item Für eine beliebige Teilmenge $M\subset\cal T$ gilt $\bigcup_{U\in M}
U\subset \cal T$.
\end{enumerate}
Die Elemente von $\cal T$ heißen \emph{offene Teilmengen} von $X$ und die
Elemente von $X$ heißen \emph{Punkte}. Ein \emph{topologischer Raum} ist ein
Paar $(X,\cal T)$ aus einer Menge $X$ und einer Topologie $\cal T$ auf $X$.
\end{definition}
\begin{definition}
Seien $(X,\cal T_X)$ und $(Y,\cal T_Y)$ topologische Räume. Eine Abbildung
$f\colon X\to Y$ heißt \emph{stetig}, wenn für jedes $U\in\cal T_Y$ das Urbild
$f^{-1}(U)\subset X$ offen ist, \ddh~$f^{-1}(U)\in\cal T_X$.
\end{definition}

Unsere Beispiele von metrischen Räumen liefern sofort Beispiel von topologischen
Räumen. Wir betrachten zuerst $\RR$. Eine Teilmenge $U\subset\RR$ heißt dann
offen, wenn für jedes $x\in U$ ein $\varepsilon > 0$ existiert, so dass
$B_\varepsilon(x) = (x-\varepsilon,x+\varepsilon)\subset U$. Die Axiome sind
erfüllt:
\begin{enumerate}
\item Für $\emptyset$ gibt es nichts zu zeigen. Für jedes $x\in \RR$ ist
natürlich $(x-\varepsilon,x+\varepsilon)\subset \RR$ für jedes beliebige
$\varepsilon>0$.
\item Sind $U\subset\RR$ und $V\subset\RR$ offen, und $x\in U\cap V$, so gibt
es ein $\varepsilon_U>0$ mit $(x-\varepsilon_U,x+\varepsilon_U)\subset U$ und ein $\varepsilon_V
> 0$ mit $(x-\varepsilon_V,x+\varepsilon_V)\subset V$. Aber dann ist
\[
B_{\min\{\varepsilon_U,\varepsilon_V\}}(x)\subset
(x-\varepsilon_U,x+\varepsilon_U)\cap (x-\varepsilon_V,x+\varepsilon_V)\subset U\cap V.
\]
\item Ist $\{U_i : i\in I\}$ eine Familie von offenen Teilmengen in $\RR$ und
$x\in \bigcup_{i\in I} U_i$, so gibt es ein $j\in I$ mit $x\in U_j$. Aber $U_j$
ist offen, also gibt es ein $\varepsilon>0$ mit
$(x-\varepsilon,x+\varepsilon)\subset U_j$. Also ist dann auch
\[
B_\varepsilon(x) = (x-\varepsilon,x+\varepsilon)\subset U_j\subset
\bigcup_{i\in I} U_i.
\]
\end{enumerate}
Ganz ähnlich zeigt man, dass $\RR^n$ mit der von einer Norm induzierten Metrik
einen topologischen Raum definiert. Wieder heißt nämlich eine Teilmenge
$U\subset\RR^n$ offen, wenn für jedes $x\in U$ ein $\varepsilon>0$ existiert,
so dass $B_\varepsilon(x) = \{x'\in\RR^n : \|x-x'\|<\varepsilon\}\subset U$.

Allgemeiner definiert jede Metrik $d$ auf einer Menge $X$ eine Topologie. Sie
heißt die von $d$ induzierte \emph{metrische Topologie} auf $X$: Eine Teilmenge
$U\subset X$ ist offen, wenn für jedes $x\in U$ ein $\varepsilon>0$ existiert
mit $B_\varepsilon(x)\subset U$. Wieder sind die Axiome erfüllt:
\begin{enumerate}
\item Für $\emptyset$ gibt es nichts zu zeigen. Für jedes $x\in X$ ist natürlich
$B_\varepsilon(x)\subset X$ für jedes beliebige $\varepsilon>0$.
\item Sind $U\subset X$ und $V\subset X$ offen und $x\in U\cap V$, so gibt es
$\varepsilon_U>0$ und $\varepsilon_V>0$ mit $B_{\varepsilon_U}(x)\subset U$ und
$B_{\varepsilon_V}(x)\subset V$. Aber dann ist
\[
B_{\min\{\varepsilon_U,\varepsilon_V\}}(x)\subset B_{\varepsilon_U}(x)\cap
B_{\varepsilon_V}(x)\subset U\cap V.
\]
\item Ist $\{U_i : i\in I\}$ eine Familie von offenen Teilmengen von $X$ und
$x\in \bigcup_{i\in I} U_i$, so gibt es ein $j\in I$ mit $x\in U_j$. Da $U_j$
offen ist, gibt es ein $\varepsilon>0$ mit $B_{\varepsilon}(x)\subset U_j$ und
damit
\[
B_\varepsilon(x)\subset U_j\subset\bigcup_{i\in I} U_i.
\]
\end{enumerate}

Hier noch einige abstraktere Beispiele für topologische Räume: Jede Menge $X$
kann mit der \emph{trivialen Topologie}\footnote{Oder der \emph{indiskreten
Topologie} oder der \emph{Klumpentopologie}} $\{\emptyset, X\}$ versehen
werden. Genauso kann jede Menge $X$ mit der \emph{diskreten Topologie}
$\powerset{X}$, der Potenzmenge von $X$, versehen werden. Für beide sind
die Axiome klar.

Seien verschachtelte topologische Räume $U_1\subset U_2\subset\dots$, so dass
die Inklusionen $\iota_{i,j}\colon U_i\into U_j$ stetig sind, gegeben. Letzteres
heißt, dass wann immer $V\subset U_j$ offen ist, so ist auch
$\iota_{i,j}^{-1}(U_j) = V\cap U_i$ offen. In dieser Situation trägt die
Vereinigung $U = \bigcup_{i=1}^\infty U_i$ eine Topologie, genannt die
\emph{finale Topologie}. In ihr ist eine Teilmenge $V\subset U$ genau dann
offen, wenn $V\cap U_i$ in $U_i$ für alle $i$ offen ist. Wir überprüfen die
Axiome:
\begin{enumerate}
\item Natürlich sind $\emptyset$ und $U$ selbst offen.
\item Seien $V,W\subset U$ offen. Dann ist $(V\cap W)\cap U_i = (V\cap
U_i)\cap(W\cap U_i)$ und letzteres ist ein endlicher Schnitt offener Mengen in
$U_i$. Also ist auch $V\cap W$ offen in $U$.
\item Ist $\{V_j\}_{j\in J}$ eine Familie offener Mengen in $U$, so ist wieder
\[
U_i\cap \bigcup_{j\in J} V_j = \bigcup_{j\in J} U_i\cap V_j
\]
eine Vereinigung von offenen Teilmengen von $U_i$. Also ist auch $\bigcup_{j\in
J} V_j$ offen in $U$.
\end{enumerate}

\begin{definition}
Sei $(X,\cal T)$ ein topologischer Raum. Eine Teilmenge $A\subset X$ heißt
\emph{abgeschlossen}, wenn $\compl{A} = X\setminus A\in\cal T$.
\end{definition}

Der Begriff \enquote{abgeschlossen} hat nichts mit \enquote{nicht offen} zu tun!
Zum Beispiel sind in jedem topologischen Raum $\emptyset$ und $X$ sowohl
abgeschlossen als auch offen.

\begin{theorem}
Eine Funktion $f\colon X\to Y$ zwischen topologischen Räumen ist genau dann
stetig, wenn für alle abgeschlossenen $A\subset Y$ das Urbild $f^{-1}(A)$ in $X$
abgeschlossen ist.
\end{theorem}
\begin{proof}
Für jede Menge $A\subset Y$ ist $f^{-1}(\compl{A}) = f^{-1}(Y\setminus A) =
X\setminus f^{-1}(A)$.
\end{proof}

\begin{definition}
Gegebenen einen topologischen Raum $X$ und eine Teilmenge $M\subset X$,
definiere
\begin{align*}
\cl{M} &= \bigcap_{\substack{A\supset M\\
\mathclap{\textnormal{abgeschlossen}}}}
A,\qquad\textnormal{den \emph{Abschluss} von $M$ in $X$,} \\
\Int{M} &= \bigcup_{\substack{U\subset M\\ \mathclap{\textnormal{offen}}}}
U,\qquad\textnormal{das \emph{Innere} von $M$} \\
\shortintertext{und}
\Bdry{M} &= \cl{M}\setminus\Int{M},\qquad\textnormal{den \emph{Rand} von $M$.}
\end{align*}
\end{definition}

Zum Beispiel ist für $M=[0,1)\subset\RR$ in der euklidischen Topologie der
Abschluss $\cl{M} = [0,1]$, das Innere $\Int{M} = (0,1)$ und der Rand $\Bdry{M}
= \{0,1\}$. Für $M=\QQ$ haben wir den Abschluss $\cl{\QQ} = \RR$ und das Innere
$\Int{\QQ} = \emptyset$ und damit auch den Rand $\Bdry{\QQ} = \RR$. Allgemein
heißt eine Teilmenge $M\subset X$ in einem topologischen Raum $X$ \emph{dicht},
wenn $\cl{M} = X$.

\subsection{Basen für Topologien}
\begin{definition}
Eine \emph{Basis für eine Topologie} auf $X$ ist eine Familie $\cal
B\subset\powerset{X}$ mit:
\begin{enumerate}
\item $\bigcup\cal B = X$, \ddh~$\cal B$ überdeckt $X$,
\item für je zwei $U,U'\in\cal B$ und $x\in U\cap U'$ gibt es ein $U''\in\cal
B$ mit $x\in U''\subset U\cap U'$.
\end{enumerate}
Erfüllt $\cal B$ nur die erste Bedingung, so ist $\cal B$ eine \emph{Subbasis
für eine Topologie} auf $X$.
\end{definition}

\begin{theorem}\label{thm:subbase-base}
Sei $\cal S$ eine Subbasis für eine Topologie auf $X$. Dann bildet die Menge
\[
\cal B = \{S_1\cap\dots\cap S_n : \textnormal{$n\in N$ und $S_1,\dots,S_n\in\cal S$}\}
\]
aller endlichen Schnitte von Mengen in $\cal S$ eine Basis für eine Topologie
auf $X$.
\end{theorem}
\begin{proof}
Nachdem $\bigcup \cal S = X$, überdeckt natürlich auch $\cal B$ ganz $X$. Seien
weiter $B = S_1 \cap \dots \cap S_n$ und $B' = S'_1 \cap\dots\cap S'_n$ Elemente
von $\cal B$. Dann ist
\[
B\cap B' = S_1\cap \dots\cap S_n\cap S'_1\cap\dots\cap S'_n\in\cal B.
\]
Insbesondere gibt es für jedes $x\in B\cap B'$ ein $B''\in\cal B$ mit $x\in
B''\subset B\cap B'$, nämlich etwa $B'' = B\cap B'$.
\end{proof}

\begin{definition}
Gegeben eine Basis oder Subbasis $\cal B$ für eine Topologie auf $X$ ist die von $\cal B$
\emph{erzeugte Topologie} $\cal T_{\cal B}$ die kleinste Topologie auf $X$, die alle Mengen in
$\cal B$ enthält.
\end{definition}
\begin{theorem}
Für eine Basis $\cal B$ ist die erzeugte Topoologie $\cal T_{\cal B}$ gegeben
durch
\[
\cal T = \left\{\bigcup_{i\in I} B_i : \text{$I$ beliebig und $B_i\in\cal B$
für alle $i\in I$}\right\}.
\]
\end{theorem}
\begin{proof}
Zuerst ist für jede Familie $\{B_i\}_{i\in I}$ mit $B_i\in\cal B$ natürlich
$\bigcup_{i\in I} B_i\in\cal T_{\cal B}$. Das heißt, wir haben die Inklusion
$\cal T\subset\cal T_{\cal B}$. Nachdem $\cal T_{\cal B}$ aber die kleinste
Topologie ist, die $\cal B$ enthält, und $\cal T$ ebenso $\cal B$ enthält,
genügt es damit zu zeigen, dass $\cal T$ bereits eine Topologie ist.

Dafür ist zunächst $\emptyset\in\cal T$ und $X\in\cal T$, denn es ist $\emptyset
= \bigcup\emptyset$ und, da $\cal B$ eine Basis für eine Topologie ist,
$\bigcup_{U\in \cal B} U = X$.

Sei nun $U = \bigcup_{i\in I} U_i$ und $V = \bigcup_{j\in J} V_j$ mit
$U_i,V_j\in\cal B$. Sei weiter $x\in U\cap V$, \ddh~es gibt ein $i\in I$ und ein
$j\in J$ mit $x\in U_i\cap V_j$. Da $\cal B$ eine Basis für eine Topologie auf
$X$ ist, gibt es dann ein $W_x\in\cal B$ mit $x\in W_x\subset U_i\cap V_j$. Aber
dann ist $U\cap V = \bigcup_{x\in U\cap V} W_x\in\cal T$.

Ist weiter $\{U_i\}_{i\in I}$ eine Familie von Teilmenen von $X$ mit
$U_i\in\cal T$, so können wir $U_i = \bigcup_{j\in J_i} B_j$ mit $B_j\in\cal B$
schreiben. Aber dann ist
\[
\bigcup_{i\in I} U_i = \bigcup_{i\in I}\bigcup_{j\in J_i} B_j
\]
eine Vereinigung von Mengen in $\cal B$ und deshalb in $\cal T$.
\end{proof}

Ist $\cal B$ eine Basis für eine Topologie auf $X$  und $\cal T$ eine Topologie
auf $X$, so nennt man $\cal B$ ein Basis für $\cal T$, wenn $\cal T = \cal T_{\cal B}$.
Zum Beispiel können wir, gegeben eine Metrik $d\colon X\times X\to\RR$, eine
Basis für die metrische Topologie finden: Sei
\[
\cal B = \{B_r(x) : x\in X, r>0\}.
\]
Dann ist $\bigcup\cal B = X$, denn für jedes $x\in X$ ist $x\in B_r(x)$ für
alle $r>0$. Ist weiter $z\in B_\varepsilon(x)\cap B_\delta(y)$, so ist
$d(z,x)<\varepsilon$ und $d(z,y)<\delta$. Setze $r = \min\{\varepsilon-d(z,x),
\delta - d(z,y)\}$. Mit dieser Wahl ist $B_r(z)\subset B_\varepsilon(x)\cap
B_\delta(y)$: Für jedes $p\in B_r(z)$ ist
\begin{align*}
d(p,x) &\leq d(p,z) + d(z,x) < r + d(z,x) \\
&\leq \varepsilon - d(z,x) + d(z,x) = \varepsilon \\
\shortintertext{und} d(p,y) &\leq d(p,z) + d(z,y) < r + d(z,y) \\
&\leq \delta - d(z,y) + d(z,y) = \delta.
\end{align*}
Also ist $\cal B$ tatsächlich eine Basis für eine Topologie auf $X$. Dass die
von $\cal B$ erzeugte Topologie genau die metrische Topologie ist, folgt aus dem
nächsten Satz.
\begin{theorem}\label{thm:gentop-char}
Sei $\cal B$ eine Basis für eine Topologie auf einer Menge $X$. Für eine
Teilmenge $U\subset X$ sind dann äquivalent:
\begin{enumerate}
\item $U\in\cal T_{\cal B}$
\item Für jedes $x\in U$ gibt es ein $V\in \cal B$ mit $x\in V\subset U$.
\end{enumerate}
\end{theorem}
\begin{proof}
Sei zuerst $U\in\cal T_{\cal B}$, etwa $U = \bigcup_{i\in I} U_i$ mit
$U_i\in\cal B$. Aber das heißt, dass es für jedes $x\in U$ ein $i\in I$ gibt,
mit $x\in U_i\subset U$.

Sei andererseits für jedes $x\in U$ ein $V_x\in\cal B$ mit $x\in V_x\subset U$
gewählt. Dann ist $U = \bigcup_{x\in U} V_x$.
\end{proof}

Mithilfe von Basen können wir Stetigkeit für Funktionen zwischen topologischen
Räumen so umformulieren, dass die Bedingung der ursprünglichen
$\varepsilon$-$\delta$-Definition für metrische Räume ähnelt.
\begin{theorem}\label{thm:continuity}
Sei $f\colon X\to Y$ eine Funktion und $\cal B_X$ eine Basis für eine Topologie
auf $X$ und $\cal B_Y$ eine Basis für eine Topologie auf $Y$. Dann sind die
folgenden Aussagen äquivalent:
\begin{enumerate}
\item $f$ ist stetig bezüglich $\cal T_{\cal B_X}$ und $\cal T_{\cal B_Y}$.
\item Für aller $U\in\cal B_Y$ ist $f^{-1}(U)\in\cal T_{\cal B_X}$.
\item Für jedes $x\in X$ und jedes $U\in\cal B_Y$ mit $f(x)\in U$ existiert ein
$V\in\cal B_X$ mit $x\in V$ und $f(V)\subset U$.
\end{enumerate}
\end{theorem}
\begin{proof}
Die Richtung (i$\Rightarrow$ii) ist klar. Für (ii$\Rightarrow$iii) sei $U\in\cal
B_Y$ mit $f(x)\in U$. Dann ist $f^{-1}(U)$ offen in $\cal T_{\cal B_X}$ und
$x\in f^{-1}(U)$. Also gibt es nach \autoref{thm:gentop-char} ein $V\in\cal B_X$
mit $x\in V\subset f^{-1}(U)$, \ddh~$f(V)\subset U$. Das ist aber genau (iii).

Für (iii$\Rightarrow$i) sei $U\subset Y$ offen in $\cal T_{\cal B_Y}$ und $x\in
f^{-1}(U)$, \ddh~$f(x)\in U$. Es gibt also nach \autoref{thm:gentop-char} ein
$U'\subset U$ mit $f(x)\in U'$ und $U'\in\cal B_Y$. Wegen (iii) gibt es dann ein
$V\subset X$ mit $V\in\cal B_X$, $x\in V$ und $f(V)\subset U'$, \ddh~$V\subset
f^{-1}(U')\subset f^{-1}(U)$. Aber wieder nach \autoref{thm:gentop-char} genügt
das, um zu sehen, dass $f^{-1}(U)\in\cal T_{\cal B_X}$.
\end{proof}

Mit \autoref{thm:gentop-char} können wir einen topologischen Beweis für
\autoref{cor:continuity-equiv-norms} geben. Insbesondere sehen wir, dass die
euklidische Topologie auf $\RR^n$ tatsächlich den Begriff der Stetigkeit
charakterisiert, unabhängig von der gewählten Norm.
\begin{theorem}
Sei $V$ ein $\RR$-Vektorraum und $\|\_\|$, $\|\_\|'$ zwei äquivalente Normen
auf $V$. Dann sind die entsprechenden metrischen Topologien gleich: $\cal
T_{\|\_\|} = \cal T_{\|\_\|'}$.
\end{theorem}
\begin{proof}
Da $\|\_\|$ und $\|\_\|'$ äquivalent sind, seien Konstanten $c,C>0$ mit
$c\|x\|\leq \|x\|' \leq C\|x\|$ für alle $x\in V$ gegeben. Schreiben wir
\begin{align*}
B_\varepsilon(x) &= \{y\in V : \|x-y\|<\varepsilon\} \\
\shortintertext{und} B'_\varepsilon(x) &= \{y\in V : \|x-y\|'<\varepsilon\},
\end{align*}
so bilden $\{B_\varepsilon(x) : \textnormal{$x\in V$, $\varepsilon>0$}\}$ und
$\{B'_\varepsilon(x) : \textnormal{$x\in V$, $\varepsilon>0$}\}$ Basen für $\cal T_{\|\_\|}$
beziehungsweise $\cal T_{\|\_\|'}$.
\begin{enumerate}
\item Jeder Ball $B_\varepsilon'(x)$ ist offen in $\cal T_{\|\_\|}$: Sei $y\in
B_\varepsilon'(x)$ und $\delta = (\varepsilon - \|y-x\|')/C$. Für $z\in
B_\delta(y)$ ist dann
\[
\|z-x\|'\leq \|z -y\|' + \|y-x\|' \leq C\|z-y\| + \|y-x\|' < C\delta + \|y-x\|'
= \varepsilon,
\]
also $y\in B_\delta(y)\subset B_\varepsilon'(x)$.
\item Sei $U\in\cal T_{\|\_\|}$ und $x\in U$. Dann existiert ein $\varepsilon>0$
mit $B_\varepsilon(x)\subset U$. Setze $\delta = c\varepsilon$. Für $y\in
B_\delta'(x)$ ist dann
\[
\|y-x\| \leq \frac{1}{c}\|y-x\|' < \frac{\delta}{c} = \varepsilon,
\]
also $y\in B_\varepsilon(x)$. Es folgt also, dass $B_\delta'(x)\subset
B_\varepsilon(x)\subset U$, und insgesamt nach \autoref{thm:gentop-char}, dass
$\cal T_{\|\_\|} = \cal T_{\|\_\|'}$.\qedhere
\end{enumerate}
\end{proof}

\begin{definition}
Ein topologischer Raum $X$ erfüllt das \emph{zweite Abzählbarkeitsaxiom}, wenn die
Topologie auf $X$ von einer höchstens abzählbar unendlichen Basis erzeugt wird.
\end{definition}

\subsection{Weitere Eigenschaften stetiger Funktionen}

Wie in metrischen Räumen können wir im Allgemeinen Folgen und ihre Konvergenz
betrachten. Allerdings ist der Begriff in allgemeinen topologischen Räumen weit
weniger hilfreich, wie wir bald sehen werden.

\begin{definition}
Eine Folge $\{x_n\}_{n\in\NN}$ in einem topologischen Raum $X$
\emph{konvergiert gegen $x\in X$}, in Symbolen $x_n\cto x$, falls für jede
offene Menge $U\subset X$ mit $x\in U$ alle bis auf endlich viele der $x_n$ in
$U$ liegen.
\end{definition}

Im Gegensatz zu unserer Erfahrung in metrischen Räumen muss der Grenzwert einer
konvergenten Folge einem allgemeinen topologischen Raum nicht eindeutig bestimmt
sein. Sei zum Beispiel $X = \{0,1\}$ mit der Topologie $\{\emptyset, X, \{0\}\}$
und betrachte die konstante Folge $x_n = 0$ für alle $n\in\NN$. Dann gilt
offenbar $x_n\cto 0$, aber $\{x_n\}_n$ konvergiert auch gegen $1$! Die einzige
offene Menge in $X$, die $1$ enthält, ist nämlich der ganze Raum $X$.

Nichtsdestotrotz lassen sich einige Sätze über konvergente Folgen für allgemeine
topologische Räume übertragen. Zum Beispiel ließen sich in metrischen Räume
stetige Funktionen als genau die folgenstetigen Funktionen charakterisieren.

\begin{theorem}
Stetige Funktionen sind \emph{folgenstetig}: Wenn $x_n\cto x$ in $X$ und
$f\colon X\to Y$ stetig ist, dann ist $f(x_n)\cto f(x)$ in $Y$.
\end{theorem}
\begin{proof}
Sei $U\subset Y$ offen mit $f(x)\in U$. Dann ist $f^{-1}(U)$ offen und $x\in
f^{-1}(U)$. Weil $x_n\cto x$ liegen dann alle bis auf endlich viele der $x_n$ in
$f^{-1}(U)$ und damit auch alle bis auf endlich viele der $f(x_n)$ ind $U$.
\end{proof}

In allgemeinen topologischen Räumen ist der Umkehrschluss aber falsch! Sei zum
Beispiel $X$ eine überabzählbar unendliche Menge. Man definierte die
\emph{ko-abzählbare Topologie} auf $X$, indem man eine Menge $U\subset X$ offen
nennt, wenn entweder $U = \emptyset$ oder $X\setminus U$ höchstens abzählbar
unendlich ist. Sei $\{p_n\}_{n\in\NN}$ eine Folge in $X$, die nicht schließlich
konstant ist, \ddh~es gibt kein $p\in X$ und $N\in\NN$ mit $p_n = p$ für alle
$n\geq N$. Wir zeigen, dass dann $\{p_n\}_n$ nicht konvergent sein kann. Sei
dafür $p\in X$ und setze $U = X\setminus\{p_n : \textnormal{$p_n\neq p$,
$n\in\NN$}\}$. Da $X$ die ko-abzählbare Topologie trägt ist dann $U$ offen und
$p\in U$. Außerdem gibt es für jedes $N\in\NN$ ein $n\geq N$ mit $p_n\neq p$,
\ddh~$p_n\not\in U$, denn ansonsten wäre $\{p_n\}_n$ schließlich konstant gleich
$p$. Insbesondere kann $\{p_n\}_n$ nicht gegen $p$ konvergieren.

Aber $p\in X$ war beliebig gewählt, also ist keine Folge $\{x_n\}_n$ in $X$ konvergent,
außer $\{x_n\}_n$ ist schließlich konstant. Natürlich ist jede schließlich
konstante Folge in jedem topologischen Raum konvergent. Das bedeutet, dass jede
Funktion $f\colon X\to Y$ folgenstetig ist, denn schließlich konstante Folgen
werden immer auf schließlich konstante Folgen abgebildet. Hingegen ist es nicht
schwer eine Funktion auf $X$ zu konstruieren, die nicht stetig ist. Zum Beispiel
ist die Identitätsabbildung aufgefasst als Funktion von $X$ mit der
ko-abzählbaren Topologie nach $X$ mit der diskreten Topologie nicht stetig: Da
$X$ überabzählbar unendlich ist, muss es eine Menge in $X$ geben, die nicht
offen ist.

Um ein Kriterium an einen topologischen Raum $X$ zu finden, unter dem folgenstetige Funktionen
$X\to Y$ automatisch stetig sind, führen wir zuerst die folgende Variante von
Basen für eine Topologie ein.

\begin{definition}
Eine Familie von offenen Mengen $\cal U$, die allen einen Punkt $x\in X$
enthalten, heißt \emph{Umgebungsbasis} für $x\in X$, falls für jede offene Menge
$V\subset X$ mit $x\in V$ ein $U\in\cal U$ existiert mit $x\in U\subset V$.
\end{definition}

Zum Beispiel ist die Familie $\{B_\varepsilon(x) : \varepsilon>0\}$ für einen
Punkt $x\in X$ in einem metrischen Raum $X$ eine Umgebungsbasis. Oder
allgemeiner ist, gegeben eine Basis $\cal B$ für eine Topologie auf einer Menge
$X$, die Familie $\{U\in\cal B : x\in U\}$ eine Umgebungsbasis für $x\in X$
bezüglich der erzeugten Topologie $\cal T_{\cal B}$.

\begin{definition}
Ein topologischer Raum $X$ erfüllt das \emph{erste Abzählbarkeitsaxiom} wenn
jedes $x\in X$ eine höchstens abzählbar unendliche Umgebungsbasis hat.
\end{definition}

\begin{theorem}
Angenommen $X$ erfüllt das erste Abzählbarkeitsaxiom. Dann ist jede
folgenstetige Funktion $f\colon X\to Y$ stetig.
\end{theorem}
\begin{proof}
Sei $x\in X$ und $f(x)\in U$ mit $U\subset Y$ offen. Dann ist $x\in f^{-1}(U)$
und es gibt eine höchstens abzählbar unendliche Umgebungsbasis für $x$. Wir
können diese Umgebungsbasis $\{V_i : i\in\NN\}$ so wählen, dass
\[
V_0\supset V_1\supset V_2\supset\cdots\ni x,
\]
denn ist $\{W_i : i\in\NN\}$ eine beliebige, höchstens abzählbare Umgebungsbasis
für $x$, setze $V_i = \bigcap_{j\leq i} W_j$. Angenommen es wäre möglich, dass
 $V_n\not\subset f^{-1}(U)$ für alle $n\in\NN$. Wählt man dann $x_n\in
V_n\setminus f^{-1}(U)$ für $n\in\NN$, erhält man eine Folge $\{x_n\}_n$, die
gegen $x$ konvergiert. Da $f$ folgenstetig ist, konvergiert dann auch
$\{f(x_n)\}$ gegen $f(x)$. Da $U$ offen ist, gibt es dann ein $N\in\NN$,
so dass $f(x_n)\in U$ für alle $n\geq N$. Insbesondere ist also $x_N\in
f^{-1}(U)$, obwohl wir $x_N\in V_N\setminus f^{-1}(U)$ gewählt hatten.

Es folgt also, dass es eine offene Menge $V_N$ gibt mit $x\in V_N\subset
f^{-1}(U)$. Da $x$ beliebig gewählt war, muss damit nach \autoref{thm:continuity}
die Funktion $f$ stetig sein.
\end{proof}

\begin{theorem}
Gegeben topologische Räume $X$, $Y$ und $Z$ mit stetigen Funktionen $f\colon
X\to Y$ und $g\colon Y\to Z$, ist die Komposition $g\circ f\colon X\to Z$
stetig.
\end{theorem}
\begin{proof}
Wenn $U\subset Z$ offen ist, dann auch
\[
(g\circ f)^{-1}(U) = f^{-1}(g^{-1}(U))
\]
also Urbild der offenen Menge $g^{-1}(U)$ unter $f$.
\end{proof}
\begin{definition}
Sei $(X,\cal T)$ ein topologischer Raum und $Y\subset X$ eine beliebige
Teilmenge. Dann ist die $\cal T$ induziert \emph{Teilraumtopologie} auf $Y$ die
Familie
\[
\cal T|_Y \coloneqq \{U\cap Y : U\in\cal T\}.
\]
\end{definition}
Dass $\cal T|_Y$ tatsächlich eine Topologie bildet überlassen wir dem Leser zur
Übung. Für $[0,1)\subset\RR$ ist in der euklidischen Topologie
\[
\cl{[0,1)} = [0,1],\quad \Int{[0,1)} = (0,1),\quad \Bdry{[0,1)} = \{0,1\},
\]
aber bezüglich der induzierten Teilraumtopologie auf $[0,1)$ haben wir
\[
\cl{[0,1)} = [0.1),\quad \Int{[0,1)} = [0,1),\quad \Bdry([0,1)) = \emptyset,
\]
da $[0,1)$ in der Teilraumtopologie natürlich offen und abgeschlossen ist.
Insbesondere sehen wir, dass eine Teilmenge $U\subset A\subset X$, die bezüglich
der Teilraumtopologie auf $A$ offen ist, nicht notwendigerweise in $X$ offen
sein muss.

\begin{theorem}
Seien $X$ und $Y$ topologische Räume und $A\subset Y$. Dann ist die Inklusion
$\iota\colon A\into Y$ stetig bezüglich der Teilraumtopologie auf $A$. Weiter
ist eine Funktion $f\colon X\to A$ genau dann stetig, wenn die Komposition
$\iota\circ f\colon X\to A\into Y$ stetig ist.
\end{theorem}
\begin{proof}
Eine Teilmenge $V\subset A$ ist genau dann offen, wenn es eine in $Y$ offene
Menge $U$ gibt mit $V = U\cap A = \iota^{-1}(U)$. Insbesondere ist in diesem
Fall $f^{-1}(V) = (\iota\circ f)^{-1}(U)$ offen.
\end{proof}

Betrachten wir als Beispiel für eine stetige Funktion die Projektion
$\pi_1\colon \RR^2\to\RR$ mit $\pi_1(x,y) = x$. Um direkt zu zeigen, dass
$\pi_1$ stetig ist, genügt es nach \autoref{thm:continuity} zu zeigen, dass für
jedes $x\in \RR$ und $\varepsilon>0$ der Zylinder
$\pi_1^{-1}((x-\varepsilon,x+\varepsilon))$ in $\RR^2$ offen ist. Sei dafür
$(a,b)\in\pi_1^{-1}((x-\varepsilon,x+\varepsilon))$ und setze $\delta =
\varepsilon - |x-a|$. Für $(c,d)\in B_\delta((a,b))$ haben wir dann
\begin{align*}
|c-x| &\leq |c-a| + |x-a| \leq \sqrt{|c-a|^2 + |d-b|^2} + |x-a| < \\
&< \delta + |x-a| = \varepsilon,
\end{align*}
also $(c,d)\in(x-\varepsilon,x+\varepsilon)\times\RR =
\pi_1^{-1}((x-\varepsilon,x+\varepsilon))$. Das genügt um zu sehen, dass
$\pi_1^{-1}((x-\varepsilon,x+\varepsilon))$ in $\RR^2$ offen ist. Später werden
wir die so genannte Produkttopologie auf $\RR^2$ definieren und sehen, dass sie
gleich der metrischen Topologie ist. Damit werden wir sofort sehen können, dass
$\pi_1$ stetig sein muss.

Andere Beispiele für stetig Funktionen findet man leicht. Sei $X$ ein beliebiger
topologischer Raum, $f\colon X\to Y$ eine Abbildung und wähle auf $Y$
die triviale Topologie. Dann ist $f$ stetig, denn $f^{-1}(\emptyset)=\emptyset$
und $f^{-1}(Y) = X$ sind beide offen. Trägt $X$ hingegen die diskrete Topologie
und $Y$ ist ein beliebiger topologischer Raum, so ist auch jede Abbildung
$f\colon X\to Y$ stetig: jede Teilmenge von $X$ ist offen, und damit insbesondere
auch $f^{-1}(U)$ für eine offene Teilmenge $U\subset Y$.

Nachdem wir gesehen haben, dass unser topologischer Begriff von Stetigkeit mit
dem vorherigen Begriff zwischen metrischen Räumen übereinstimmt, erhalten wir
sofort alle stetigen Funktionen zwischen metrischen Räumen als Beispiele
stetiger Funktionen. Konkreter ist etwa die Funktion $\tanh\colon \RR\to (-1,1)$ bezüglich
der euklidischen Topologie stetig. Besser noch, $\tanh$ hat eine inverse
Funktion $\atanh\colon (-1,1)\to\RR$, die ebenso stetig ist. Man sagt, $\tanh$
ist ein Homöomorphismus zwischen $\RR$ und $(-1,1)$.
\begin{definition}
Eine stetige Abbildung $f\colon X\to Y$ heißt \emph{Homöomorphismus}, wenn eine
weitere stetige Abbildung $g\colon Y\to X$ existiert mit $g\circ f = \id_X$ und
$f\circ g = \id_Y$.
\end{definition}
Das Thema dieses Kurses wird sein, Methoden kennenzulernen, mit denen man
erkennen kann ob zwei topologische Räume homöomorph sein können. Als erstes
Beispiel betrachte die Exponentialfunktion $e^{i\_}\colon [0,2\pi)\to S^1 = \{z
: |z| = 1\}\subset\CC$. Der Einheitskreis $S^1$ trägt hier wie das halboffene
Intervall $[0,2\pi)$ die Teilraumtopologie. Diese Abbildung ist stetig und
bijektiv, es gibt also eine Umkehrabbildung $\arg\colon S^1\to{} [0,2\pi)$. Diese
Umkehrabbildung ist aber nicht stetig! Zum Beispiel ist $\arg^{-1}([0,\pi))$
nicht offen in $S^1$, obwohl $[0,\pi) = (-1,\pi)\cap [0,2\pi)$ in $[0,2\pi)$
offen ist: Jeder offene Ball um $1 = e^{i0}\in S^1$ enthält ein $z\in S^1$ mit
$\Im(z)<0$, aber $\arg^{-1}([0,\pi))$ enthält nur solche $z\in S^1$ mit
$\Im(z)\geq 0$.

Wir werden später sehen, dass nicht nur $e^{i\_}$ kein Homöomorphismus zwischen
$[0,2\pi)$ und $S^1$ ist, sondern dass es überhaupt keinen solchen Homöomorphismus
geben kann.

\subsection{Konstruktionen von topologischen Räumen}

Wir haben bisher die euklidische Topologie auf $\RR^2 = \RR\times\RR$ gesehen.
Ihre offenen Teilmengen können als beliebige Vereinigungen
\[
\bigcup_{i\in I} (a_i, b_i)\times (a'_i,b'_i)
\]
geschrieben werden, \ddh~Mengen der Form $(a,b)\times (a',b')$ bilden eine Basis
für die euklidische Topologie auf $\RR^2$. Können wir analog eine Topologie
auf einem Produktraum $X\times Y$ für beliebige topologische Räume $X$ und $Y$
definieren?

Sei $X = \prod_{i\in I} X_i$ ein Produkt beliebiger topologischer Räume $X_i$
und sei $\pi_j\colon X\to X_j$ die Projektion auf den $j$-ten Faktor. Dann ist
\[
\cal S = \{ \pi_j^{-1}(U) : \textnormal{$j\in I$ und $U\subset X_j$ offen} \}
\]
eine Subbasis für eine Topologie auf $X$.
\begin{definition}
Die von $\cal S$ erzeugte Topologie auf $\prod_{i\in I} X_i$ heißt
\emph{Produkttopologie}.
\end{definition}
Man sieht leicht, dass die nach \autoref{thm:subbase-base} zu $\cal S$ gehörige
Basis aus den Mengen der Form $\prod_{i\in I} U_i$ besteht mit $U_i\subset X_i$ offen,
aber mit nur endlich vielen der $U_i$ verschieden von $X_i$. Man kann auch eine
größere Topologie auf $\prod_{i\in I} X_i$ definieren, wenn man diese letzte
Bedingung vernachlässigt. Sie heißt \emph{Boxtopologie}, erfüllt aber viele der
guten Eigenschaften der Produkttopologie nicht.

\begin{lemma}\label{lem:products}
Sei $X = \prod_{i\in I} X_i$ ein Produkt topologischer Räume $X_i$ versehen mit
der Produkttopologie. Dann:
\begin{enumerate}
\item Die Projektionen $\pi_i\colon X\to X_i$ sind stetig.
\item Die Produkttopologie ist die kleinste Topologie auf $X$ für die alle
Projektionen $\pi_i\colon X\to X_i$ stetig sind, \ddh~ist $\cal T$ eine andere
Topologie für die alle $\pi_i$ stetig sind, so ist jede in der Produkttopologie
offene Menge $U$ bereits in $\cal T$ enthalten.
\item Sei $Y$ ein topologischer Raum und für jedes $i\in I$ sei $f_i\colon Y\to
X_i$ stetig. Dann existiert genau eine stetige Funktion $g\colon Y\to X$ mit
$\pi_i\circ g = f_i$ für alle $i\in I$.
\[
\begin{tikzcd}
Y \arrow[rd, "f_i"'] \arrow[r, "g", dashed] & X \mathrlap{{}= \prod_{i\in I} X_i} \arrow[d, "\pi_i"] \\
{} & X_i
\end{tikzcd}
\]
\end{enumerate}
\end{lemma}
\begin{proof}\needspace{2\baselineskip}\leavevmode
\begin{enumerate}
\item Sei $U\subset X_i$ offen. Dann ist $\pi_i^{-1}(U)$ enthalten in der
Subbasis, die die Produkttopologie auf $X$ definiert. Insbesondere ist damit
$\pi_i^{-1}(U)$ natürlich offen.
\item Sei $\cal T$ eine Topologie auf $X$, bezüglich der alle $\pi_i\colon X\to X_i$
stetig sind, und sei $U\subset X_i$ offen. Dann ist $\pi_i^{-1}(U_i)$ offen
bezüglich der Produkttopologie und auch bezüglich $\cal T$. Das heißt, dass die
Subbasis $\cal S$, die die Produkttopologie erzeugt, in $\cal T$ enthalten ist.
Das bedeutet dann aber, dass auch die von $\cal S$ erzeugte Topologie in $\cal
T$ enthalten ist, was zu zeigen war.
\item Die Abbildung $g\colon Y\to X$ ist definiert durch $g(y) = (f_i(y))_{i\in
I}$ und durch die Bedingung $\pi_i\circ g = f_i$ für alle $i\in I$ eindeutig
bestimmt. Es bleibt zu sehen, dass $g$ stetig ist. Dafür genügt es zu zeigen,
dass die Urbilder der Elemente der Subbasis $\cal S$ in $Y$ offen sind. Sei
dafür $i\in I$ und $U\subset X_i$ offen, so dass $\pi_i^{-1}(U)\in\cal S$. Dann
ist
\[
g^{-1}(\pi_i^{-1}(U)) = (\pi_i\circ g)^{-1}(U) = f_i^{-1}(U)
\]
offen in $Y$, denn $f_i\colon Y\to X_i$ war als stetig angenommen.\qedhere
\end{enumerate}
\end{proof}

Eigenschaft (iii) \autoref{lem:products} nennt man auch die \emph{universelle Eigenschaft} der
Produkttopologie. Ein erstes Beispiel für die Produkttopologie haben wir schon
mit der euklidischen Topologie auf $\RR^n = \RR\times\dots\times \RR$ gesehen.
Man sieht leicht, dass sie tatsächlich mit der Produkttopologie übereinstimmt.
Einfache Beispiele stetiger Funktionen bezüglich der Produkttopologie sind etwa
die Addition $+\colon\RR\times\RR\to\RR$ und die Multiplikation
$\cdot\colon\RR\times\RR\to\RR$.

Genauso trägt die Menge $\Mat_m(\RR)$ von $m\times m$-Matrizen die
Produkttopololgie, wir fassen sie als $\RR^{m^2}$ auf. Hier gibt es ebenso
stetige Funktion $+\colon \Mat_m(\RR)\times\Mat_m(\RR)\to\Mat_m(\RR)$ und
$\cdot\colon \Mat_m(\RR)\times\Mat_m(\RR)\to\Mat_m(\RR)$. In den beiden letzten
Beispielen gibt es zusätzlich stetige Abbildungen
$(\_)^{-1}\colon\RR\setminus\{0\}\times\RR\setminus\{0\}\to\RR\setminus\{0\}$
und $(\_)^{-1}\colon\GL_m(\RR)\times\GL_m(\RR)\to\GL_m(\RR)$. Damit haben wir
auch Beispiele der folgenden Definition.
\begin{definition}
Eine \emph{topologische Gruppe} ist eine Gruppe $(G,\cdot,e)$ zusammen mit einer
Topologie auf $G$ bezüglich der die Abbildungen
\begin{align*}
G\times G\to G,\quad &(g,h)\mapsto g\cdot h \\
\shortintertext{und} G\to G,\quad &g \mapsto g^{-1}
\end{align*}
stetig sind. Dabei versteht sich $G\times G$ als mit der Produkttopologie
versehen.
\end{definition}

Wir definieren eine Äquivalenzrelation $\sim$ auf $\RR$ indem wir für $x,y\in
\RR$ genau dann $x\sim y$ schreiben, wenn $x - y\in\ZZ$. Damit können wir die
Quotientenmenge $\RR/{\sim} \eqqcolon \RR/\ZZ$ bilden. Dann können wir $\RR/\ZZ$
auch mit einer Topologie versehen, bezüglich der die Projektion $\RR\to\RR/\ZZ$,
die eine reelle Zahl $x$ auf ihre Äquivalenzklasse $[z]$ in $\RR/\ZZ$ abbildet,
stetig ist. Allgemeiner definieren wir wie folgt.
\begin{definition}\label{defn:quotients}
Sei $X$ ein topologischer Raum und $\sim$ eine Äquivalenzrelation auf $X$. Wir
bezeichnen mit $X/{\sim}$ die Menge der Äquivalenzklassen bezüglich $\sim$ und
mit $q\colon X\to X/{\sim}$ die Quotientenabbildung. Dann ist
\[
\cal T = \{ U \subset X/{\sim} : \textnormal{$q^{-1}(U)$ offen in $X$} \}
\]
eine Topologie auf $X/{\sim}$. Sie heißt die \emph{Quotiententopologie}.
\end{definition}
\begin{lemma}\label{lem:quotients}
Mit der Notation wie in \autoref{defn:quotients} ist die Quotiententopologie
$\cal T$ tatsächlich eine Topologie. Weiterhin gilt:
\begin{enumerate}
\item Die Quotientenabbildung $q\colon X\to X/{\sim}$ ist stetig.
\item Die Quotiententopologie $\cal T$ ist die größte Topologie bezüglich der $q$ stetig
ist, \ddh~ist $\cal T'$ eine weitere Topologie auf $X/{\sim}$ bezüglich der $q$
stetig ist, so ist $\cal T'\subset\cal T$.
\item Ist $f\colon X\to Y$ stetig mit $f(x) = f(y)$ für alle $x,y\in X$ mit
$x\sim y$, so existiert genau eine stetige Abbildung $g\colon X/{\sim}\to Y$ mit
$g\circ q = f$.
\[
\begin{tikzcd}
X \arrow[r,"f"] \arrow[d, "q"'] & Y \\
X/{\sim} \arrow[ur, "g"']
\end{tikzcd}
\]
\end{enumerate}
\end{lemma}
\begin{proof}
Natürlich sind $q^{-1}(\emptyset) = \emptyset$ und $q^{-1}(X/{\sim}) = X$ offen
in $X$. Sind $U, V\subset X/{\sim}$ offen in der Quotiententopologie, so ist
$q^{-1}(U\cap V) = q^{-1}(U)\cap q^{-1}(V)$ offen in $X$ und damit auch $U\cap
V$ offen im Quotienten.

Ist schließlich $\{U_i\}_{i\in I}$ eine Familie offener Mengen in $X/{\sim}$, so
ist $q^{-1}(U_i)$ offen in $X$ für jedes $i\in I$ und deshalb auch $q^{-1}(\bigcup_{i\in I} U_i) =
\bigcup_{i\in I}q^{-1}(U_i)$ offen in $X$. Das heißt, $\bigcup_{i\in I} U_i$ ist
offen im Quotienten $X/{\sim}$.
\begin{enumerate}
\item folgt direkt aus der Definition der Quotiententopologie.
\item Sei $\cal T'$ eine Topologie auf $X/{\sim}$ bezüglich der $q$ stetig ist.
Dann ist für jede Menge $U\in\cal T'$ das Urbild $q^{-1}(U)$ offen in $X$. Aber
das heißt genau, dass $U\in\cal T$. Da $U$ beliebig war, bedeutet das $\cal
T'\subset\cal T$.
\item Gegeben eine Äquivalenzklasse $[x]\in X/{\sim}$ definieren wir $g([x]) =
f(x)$. Das ist möglich, denn wäre $[x] = [y]$, \ddh~$x\sim y$, für ein weiteres
 $y\in X$, so wäre $f(x) = f(y)$ und es gibt keine Mehrdeutigkeit in dieser
Definition von $g$. Damit haben wir auch sofort, dass $(g\circ q)(x) = g([x]) =
f(x)$ für alle $x\in X$ und $g$ ist durch diese Bedingung eindeutig bestimmt. Es
bleibt zu zeigen, dass $g$ stetig ist. Sei dafür $U\subset Y$ offen. Dann ist
\[
q^{-1}(g^{-1}(U)) = (g\circ q)^{-1}(U) = f^{-1}(U)
\]
offen in $X$ und damit $g^{-1}(U)$ offen in $X/{\sim}$.\qedhere
\end{enumerate}
\end{proof}
Auch hier nennt man Eigenschaft (iii) in \autoref{lem:quotients} die
\emph{universelle Eigenschaft} der Quotiententopologie auf $X/{\sim}$. Sie
charakterisiert die Quotiententopologie bis auf Homöomorphismus.
\begin{definition}
Seien $X$ und $Y$ topologische Räume. Eine Abbildung $f\colon X\to Y$ heißt
\emph{offen}, wenn für jede offene Menge $U\subset X$ das Bild $f(U)\subset Y$
offen ist.
\end{definition}
Mithilfe dieser Definition können wir charakterisieren, wann eine stetige
Bijektion ein Homöomorphismus ist. In der Tat ist eine solche stetige Bijektion
$f\colon X\to Y$ genau dann ein Homöomorphismus, wenn sie offen ist. Denn für
alle Mengen $U\subset X$ ist $(f^{-1})^{-1}(U) = f(U)$ und damit eine bijektive Abbildung
genau dann offen, wenn ihr Inverses stetig ist.

Wir hatten die Quotiententopologie am Beispiel $\RR/\ZZ$ eingeführt, wobei in
$\RR/\ZZ$ genau die reellen Zahlen identifiziert wurden, deren Differenz eine
ganze Zahl ist. Wir wollen nun zeigen, dass $\RR/\ZZ$ mit der Quotiententopologie
homöomorph zum Einheitskreis $S^1$ ist. Man betrachte dafür die stetige Abbildung
\[
f\colon \RR\to S^1,\quad f(x) = e^{2\pi i x}.
\]
Nach der universellen Eigenschaft der Quotiententopologie gibt es eine eindeutig
bestimmte stetige Abbildung $g\colon \RR/\ZZ\to S^1$, so dass
\[
\begin{tikzcd}
\RR \arrow[r, "f"] \arrow[d, "q"'] & S^1 \\
\RR/\ZZ \arrow[ur, "g"]
\end{tikzcd}
\]
kommutiert. Diese $g$ ist surjektiv, denn $f$ war bereits surjektiv. Außerdem
ist $g$ injektiv, denn $e^{2\pi i x} = e^{2\pi i x'}$ ist genau dann der Fall,
wenn $x-x'\in\ZZ$. Das heißt, um zu sehen, dass $g$ ein Homöomorphismus ist,
genügt es zu zeigen, dass $g$ eine offene Abbildung ist. Zuerst sehen wir, dass
$f$ selbst eine offene Abbildung ist. Ist nämlich $(-\varepsilon,\varepsilon)\subset\RR$
mit $\varepsilon \leq 1/2/$ ein offenes Intervall, so sieht man leicht, dass
$f((-\varepsilon,\varepsilon))\subset S^1$ offen ist. Außerdem ist $f(x+x')
=f(x)f(x')$ und $f(0) = 1$ und damit sind auch die Bilder aller Intervalle der
Form $(x-\varepsilon,x+\varepsilon)$ mit $\varepsilon\leq 1/2$ offen in $S^1$.
Diese offenen Intervalle bilden eine Basis für die euklidische Topologie auf
$\RR$ und deshalb  sehen wir, dass $f$ eine offene
Abbildung sein muss. Ist weiter $U\subset\RR/\ZZ$ offen, so ist $g(U) = f(q^{-1}(U))$ offen. Also ist
$g$ eine offene Abbildung und nach unserer vorherigen Bemerkung folgt, dass $g$
ein Homöomorphismus ist.

Für ein weiteres Beispiel sei $X=[0,1]$ mit der von $\RR$ induzierten
Teilraumtopologie. Wir führen eine Äquivalenzrelation $\sim$ auf $X$ ein, indem
wir $x\sim x'$ genau dann schreiben, wenn entweder $x=x'$ oder $x,x'\in\{0,1\}$.
Dann ist der Quotient $X/{\sim}$ ebenfalls homöomorph zum Einheitskreis $S^1$.
Wir können nämlich unser Argument von oben mit der Einschränkung von $f$ auf
$[0,1]$ wortgleich wiederholen.

Sei nun $X = [0,1]^2$ mit der von $\RR^2$ induzierten Teilraumtopologie mit der
Äquivalenzrelation $\sim$, bezüglich der genau dann $(x,y)\sim (x',y')$ gilt, wenn
entweder $(x,y) = (x',y')$ gilt, oder $x,x'\in\{0,1\}$ mit $y=y'$, oder $x=x'$
mit $y,y'\in\{0,1\}$, oder wenn $\{x,x'\} = \{y,y'\} = \{0,1\}$. Der Quotient
$X/{\sim}$ ist dann homöomorph zum Torus $S^1\times S^1$.

Für ein etwas exotischeres Beispiel sei $X = \RR\times\{0,1\}\subset \RR^2$ mit
der von $\RR^2$ induzierten Teilraumtopologie. Wir definieren wieder eine Äquivalenzrelation
$\sim$ auf $X$, indem wir $(x,y)\sim (x'y')$ genau dann schreiben, wenn entweder
$(x,y) = (x',y')$ oder $x = x'\neq 0$. Der Quotient $X/{\sim}$ ähnelt dann der
reellen Geraden $\RR$, enthält aber \emph{zwei} Punkte, die sich wie der
Ursprung in $\RR$ verhalten. Entsprechend nennt man $X/{\sim}$ die \emph{Gerade
mit zwei Ursprüngen}.

\begin{definition}
Sei $X$ ein topologischer Raum und $A\subset X$ eine beliebige Teilmenge. Dann
definieren wir eine Äquivalenzrelation $\sim_A$ auf $X$, indem wir genau dann
$x\sim_A x'$ schreiben, wenn entweder $x=x'$ oder $x,x'\in A$. Den Quotienten
$X/{\sim_A}$ bezeichnen wir dann kurz als $X/A$.
\end{definition}

Letzere Definition haben wir bereits in einem Beispiel gesehen. Nämlich ist
$X/\{0,1\}$ homöomorph zu $S^1$. Betrachten wir andererseits etwa $X/[0,1]$, so
stellen wir fest, dass dieser Quotient wieder homöomorph zu $\RR$ ist. Wir haben
nämlich eine stetige Abbildung
\[
f\colon \RR\to\RR,\quad f(x) = \begin{cases}
x & x<0 \\
0 & x\in [0,1] \\
x-1 & x > 1
\end{cases}
\]
die die Äquivalenzrelation $\sim_{[0,1]}$ auf $\RR$ respektiert. Die universelle
Eigenschaft der Quotiententopologie liefert dann wieder eine stetige Abbildung
$g\colon \RR/[0,1]\to\RR$, so dass
\[
\begin{tikzcd}
\RR \arrow[d] \arrow[r, "f"] & \RR \\
\RR/[0,1] \arrow[ur, "g"']
\end{tikzcd}
\]
kommutiert. Dann ist $g$ augenscheinlich bijektiv und sogar ein Homöomorphismus.
Sei dafür $U\subset \RR/[0,1]$ offen in der Quotiententopologie und sei $V =
q^{-1}(U)$. Dann gibt es verschieden Fälle:
\begin{enumerate}
\item Ist $V\cap [0,1] = \emptyset$ so ist $f(V) = g(U)$ offen.
\item Ist $[0,1]\subset V$, so ist $f(V) = g(U)$ ebenso offen.
\item Ist $[0,1]\cap V\neq\emptyset$ aber $[0,1]\not\subset V$, so ist $V =
([0,1]\cap V) \cup (V \setminus [0,1])$ und man sieht ebenso leicht, dass $f(V) =
g(U)$ offen ist.
\end{enumerate}
Im Gegensatz zu $\RR/[0,1]$ ist $\RR/(0,1)$ nicht homöomorph zu $\RR$: Sei
$q\colon R\to\RR/(0,1)$ die Quotientenabbildung. Dann ist die einelementige Menge
$\{q(1/2)\}$ offen in $\RR/(0,1)$, denn das Urbild $q^{-1}(\{q(1/2)\}) = (0,1)$ ist offen
in $\RR$. Aber keine offene Menge in $\RR$ enthält nur ein Element.

\subsection{Trennungsaxiome}

\begin{definition}
Ein topologischer Raum $X$ hat die \emph{Hausdorffeigenschaft} (oder ist \emph{Hausdorff}), falls für verschiedene Punkte $x\neq y\in X$ offene Mengen $U,V\subset X$ existieren, so dass $x\in U$, $y\in V$ und $U\cap V = \emptyset$.
\end{definition}

Zum Beispiel ist $\RR$ mit der Standardtopologie Hausdorff. Seien nämlich $x,y\in\RR$ verschieden. Dann ist $\delta = |x-y| > 0$ und wir setzen $U = B_\varepsilon(x)$ und $V= B_\varepsilon(y)$ mit $\varepsilon = \delta/2$. Dann ist $U\cap V = \emptyset$ und $x\in U$ und $y\in V$.

Jede Menge $X$ mit der diskreten Topologie ist Hausdorff. Tatsächlich sind für verschiedene Punkte $x,y\in\RR$ die Mengen $\{x\}$ und $\{y\}$ offen und disjunkt. Hingegen ist die Topologie $\{\emptyset, \{a\}, \{a,b\}\}$ auf $X = \{a,b\}$ nicht Hausdorff, denn es gibt in dieser Topologie keine offene Menge die $a$ aber nicht $b$ enthält. Genauso ist $X = \{a,b\}$ mit der trivialen Topologie $\{\emptyset,\{a,b\}\}$ nicht Hausdorff.

\begin{lemma}\needspace{2\baselineskip}\leavevmode
\begin{enumerate}
\item In einem Hausdorffraum sind einpunktige Mengen abgeschlossen.
\item Teilräume und Produkträume von Hausdorffräumen sind Hausdorff.
\end{enumerate}
\end{lemma}
\begin{proof}\needspace{2\baselineskip}\leavevmode
\begin{enumerate}
\item Sei $x\in X$ und $X$ ein Hausdorffraum. Dann ist zu zeigen, dass $X\setminus\{x\}$ offen ist. Sei dafür $y\in X\setminus\{x\}$ beliebig. Dann gibt es eine offene Menge $U_y\subset X$ mit $y\in U_y$ und $x\not\in U_y$. Damit ist $X\setminus\{x\} = \bigcup_{y\in X\setminus\{x\}} U_y$ eine Vereinigung offener Mengen und daher selbst offen.
\item Seien zuerst $M\subset X$ ein Teilraum eines Hausdorffraums $X$ und $x,y\in M$ verschiedene Punkte. Dann gibt es disjunkte offene Menge $U\subset X$ und $V\subset X$ mit $x\in U$ und $y\in V$. Also ist $x\in U\cap M$ und $y\in V\cap M$ für die disjunkten, in $M$ offenen Mengen $U\cap M$ und $V\cap M$.

Angenommen $I$ ist eine beliebige Indexmenge und $\{X_i\}_{i\in I}$ ist eine Familie von Hausdorffräumen. Seien $(x_i)_{i\in I}$ und $(y_i)_{i\in I}$ verschiedene Punkte in $\prod_{i\in I}X_i$, das heißt, es gibt ein $j\in I$ mit $x_j\neq y_j$. Da $X_j$ Hausdorff ist, gibt es dann disjunkte offene Mengen $U_j,V_j\subset X_j$ mit $x_j\in U_j$ und $y_j\in V_j$. Für die Projektion $\pi\colon\prod_{i\in I} X_i\to X_j$ sind dann die Urbilder $\pi^{-1}(U_i)$ und $\pi^{-1}(V_j)$ offen und disjunkt in $\prod_{i\in I} X_i$. Außerdem ist dann $(x_i)_i\in\pi^{-1}(U_j)$ und $(y_i)_i\in\pi^{-1}(V_j)$.\qedhere
\end{enumerate}
\end{proof}

Quotienten von Hausdorffräumen müssen nicht immer Hausdorff sein. Zum Beispiel ist die Gerade mit zwei Ursprüngen $X = \RR\times\{0,1\}/{\sim}$ kein Hausdorffraum, obwohl $\RR\times\{0,1\}$ Hausdorff ist. Seien $0 = [(0,0)]$ und $0' = [(0,1)]$ die beiden Ursprünge. Dann enthält jede offene Umgebung von $0$ bzw. $0'$ eine Menge der Form $\pi((-\varepsilon,\varepsilon)\times\{0\})$ bzw. $\pi((-\varepsilon,\varepsilon)\times\{1\})$ für $\varepsilon>0$ klein genug und wobei $\pi\colon\RR\times\{0,1\}\to X$ die Quotientenabbildung bezeichnet. Aber es ist immer
\[
\pi((-\varepsilon,\varepsilon)\times\{0\})\cap\pi((-\varepsilon,\varepsilon)\times\{1\})\neq\emptyset.
\]

\begin{definition}
Ein topologischer Raum $X$ heißt \emph{\Tone}, falls jede einpunktige Menge in $X$ abgeschlossen ist.
\end{definition}

Wir haben schon gesehen, dass jeder Hausdorffraum \Tone ist. Aber die Umkehrung ist falsch! Zum Beispiel ist die Gerade mit zwei Ursprüngen \Tone aber nicht Hausdorff. Genauso ist $\ZZ$ mit der kofiniten Topologie \Tone aber nicht Hausdorff: alle endlichen Teilmengen von $\ZZ$ sind abgeschlossen in der kofiniten Topologie. Aber keine zwei nichtleeren offenen Mengen in der kofiniten Topologie auf $\ZZ$ sind disjunkt.

\begin{definition}
Sei $X$ ein \Tone-Raum.
\begin{enumerate}
\item $X$ heißt \emph{regulär}, falls es zu jedem $x\in X$ und jeder abgeschlossenen Menge $A\subset X$ mit $x\not\in A$ offene Mengen $U$ und $V$ in $X$ gibt, so dass $x\in U$, $A\subset V$ und $U\cap V = \emptyset$.
\item $X$ heißt \emph{normal}, falls es zu zwei disjunkten, abgeschlossenen Mengen $A,B\subset X$ disjunkte offene Mengen $U$ und $V$ in $X$ gibt, so dass $A\subset U$, $B\subset V$ und $U\cap V = \emptyset$.
\end{enumerate}
\end{definition}
Manche Autoren fordern nicht, dass reguläre bzw. normale Räume \Tone sind.

\begin{lemma}\label{lem:pf-urysohn}\needspace{2\baselineskip}\leavevmode
\begin{enumerate}
\item Für $X$ regulär ist jeder Teilraum $Y\subset X$ regulär.
\item Für $X$ normal ist ein Teilraum $Y\subset X$ normal, sobald $Y\subset X$ abgeschlossen ist.\proofomitted
\end{enumerate}
\end{lemma}

\begin{theorem}[Urysohn--Tietze]\label{thm:urysohn-tietze}
Sei $X$ ein \Tone-Raum. Dann sind die folgenden Aussagen äquivalent:
\begin{enumerate}
\item $X$ ist normal.
\item Zu zwei disjunkten abgeschlossenen Teilmengen $A,B\subset X$ existiert eine stetige Funktion $f\colon X\to{} [0,1]$ mit $f|_A = 0$ und $f|_B = 1$.
\item Für jede abgeschlossene Teilmenge $A\subset X$ mit einer stetige Funktion $f\colon A\to\RR$ gibt es eine stetige Fortsetzung $F\colon  X\to\RR$ mit $F|_A = f$ und \[
\sup_{x\in X} F(x) = \sup_{x\in A} f(x)\quad\textnormal{und}\quad \inf_{x\in X} F(x) = \inf_{x\in A} f(x).
\]
Ist außerdem $|f(x)| < C$ für alle $a\in A$ lásst sich $F$ so wählen, dass $|F(x)| < C$ für alle $x\in X$.
\end{enumerate}
Der Teil (i$\cto$ii) ist das Lemma von Urysohn und eine Funktion $f$ wie in (ii) heißt \emph{Urysohnfunktion}. Der Teil (i$\cto$iii) ist der Fortsetzungssatz von Tietze.
\end{theorem}
\begin{lemma}
Sei $X$ normal. Für $A\subset U$ mit $A$ abgeschlossen und $U$ offen gibt es eine offene Menge $V\subset X$, so dass $A\subset V\subset\cl{V}\subset U$.
\end{lemma}
\begin{proof}
Da $X\setminus U$ abgeschlossen, $A\cap(X\setminus U)=\emptyset$ und $X$ normal ist, gibt es disjunkte offene Mengen $V,W\subset X$, so dass $A\subset V$ und $X\setminus U\subset W$. Dann ist aber auch $V\subset X\setminus W\subset U$ und da $X\setminus W$ abgeschlossen ist, folgt $A\subset V\subset\cl{V}\subset X\setminus W\subset U$.
\end{proof}

\begin{lemma}\label{lem:claim23}
Sei $X$ ein Raum, der das Ursyohnlemma, \autoref{thm:urysohn-tietze} Teil (ii), erfüllt, und sei $f\colon A\to\RR$ stetig und beschränkt, etwa $|f(a)| \leq c$ für alle $a\in A$. Dann existiert eine stetig Abbildung $h\colon X\to\RR$ mit $|h(x)|\leq \tfrac{1}{3}c$ für $x\in X$ und $|f(a) - h(a)|\leq \tfrac{2}{3} c$ für $a\in A$.
\end{lemma}
\begin{proof}
Die Mengen
\[
A_+ = \{a\in A : f(a) \geq c/3\}\quad\textnormal{und}\quad A_- = \{a\in A : f(a) \leq -c/3\}
\]
sind abgeschlossen und disjunkt. Also gibt es eine Urysohnfunktion $h\colon X\to{} [-\tfrac{1}{3}c,\tfrac{1}{3}c]$ mit $h|_{A_+} = \tfrac{1}{3}c$ und $h|_{A_-} = -\tfrac{1}{3}c$. Für dieses $h$ sieht man leicht, dass $|f(a) - h(a)| \leq \tfrac{2}{3}c$ für alle $a\in A$.
\end{proof}

\begin{proof}[Beweis von \autoref{thm:urysohn-tietze}]
Wir zeigen zuerst (ii$\cto$i). Seien dafür $A,B\subset X$ abgeschlossen mit $A\cap B = \emptyset$. Dann gibt es eine Urysohnfunktion $f\colon X\to{}[0,1]$ mit $f|_A = 0$ und $f|_B = 1$. Aber dann sind $U = f^{-1}([0,1/2))$ und $V = f^{-1}((1/2,1])$ offen in $X$ und disjunkt. Außerdem ist $A\subset U$ und $B\subset V$.

Nun zu (iii$\cto$ii); seien $A,B\subset X$ abgeschlossen und disjunkt. Dann ist $f\colon A\cup B \to{}[0,1]$ mit $f|_A = 0$ und $f|_B = 1$ stetig und nach (iii) gibt es eine Fortsetzung $F\colon X\to{}[0,1]$. Diese Fortsetzung ist dann eine Urysohnfunktion.

Für (i$\cto$ii) sei $X$ normal mit disjunkten abgeschlossenen Teilmengen $A,B\subset X$. Da $A\subset X\setminus B$ gibt es nach \autoref{lem:pf-urysohn} eine offene Menge $O_1\subset X$ mit $A\subset O_1\subset\cl{O_1}\subset X\setminus B$. Dann gibt es wieder nach \autoref{lem:pf-urysohn} eine offene Menge $O_0\subset X$ mit
\[
A\subset O_0\subset \cl{O_0}\subset O_1\subset \cl{O_1}\subset X\setminus B.
\]
Induktiv konstruieren wir dann für jedes $n\in\NN$ und $0\leq p\leq 2^n$ offene Teilmengen $O_{p/2^n}\subset X$ mit
\[
A\subset O_{p/2^n}\subset\cl{O_{p/2^n}}\subset O_{q/2^n}\subset\cl{O_{q/2^n}}\subset B
\]
für $0\leq p \leq q \leq 2^n$. Ist nämlich $p = 2p'$ gerade, setze $O_{p/2^n} = O_{p'/2^{n-1}}$. Ist $p = 2p'+1$, gibt es nach \autoref{lem:pf-urysohn} eine offene Menge $O_{p/2^n}$ mit
\[
\cl{O_{p'/2^{n-1}}}\subset O_{p/2^n} \subset\cl{O_{p/2^n}} \subset X\setminus O_{(p'+1)/2^{n-1}}.
\]
Damit können wir eine Funktion $f\colon X\to{} [0,1]$ definieren, indem wir
\[
f(x) = \inf\left\{\tfrac{p}{2^n}\in\RR : x\in O_{p/2^n}\right\}
\]
für $x\in O_1$ und $f(x) = 1$ für $x\not\in O_1$ setzen. Die Funktion $f$ ist stetig: sei $x\in X$ beliebig und $\varepsilon > 0$. Dann gibt es $d_1 = p/2^m$ und $d_2 = q/2^n$ mit
\[
f(x) - \varepsilon < d_1 < f(x) < d_2 < f(x) + \varepsilon.
\]
Sei $U = O_{d_2}\setminus \cl{O_{d_1}}$. Dann ist $x\in U$ und $f(U)\subset B_\varepsilon(f(x))$: zuerst ist $x\in O_{d_2}$, da $f(x) < d_2$, und $x\not\in\cl{O_{d_1}}$, da $f(x) > d_1$. Für $y\in U$ gilt genauso $d_1 < f(y) < d_2$ und damit $f(y)\in B_\varepsilon(f(x))$. Insgesamt genügt das, um zu sehen, dass $f$ stetig ist.

Für (ii$\cto$iii) sei $A$ abgeschlossen und $f\colon A\to\RR$ stetig. Wir nehmen zuerst an, dass $f$ beschränkt ist, etwa $|f(a)| < c$ für alle $a\in A$. Wir konstruieren induktiv für alle $n\in\NN_0$ und $0\leq m\leq n$ eine stetige Funktion $h_m\colon X\to\RR$ mit
\[
|h_m(x)| \leq \tfrac{1}{3}\left(\tfrac{2}{3}\right)^m c\quad\textnormal{und}\quad\left|f(a) - \sum_{m=0}^n h_m(a)\right| \leq \left(\tfrac{2}{3}\right)^{n+1} c.
\]
für $x\in X$ und $a\in A$. Setze $h_0 = h$ für $h$ wie in \autoref{lem:claim23}. Für $n > 0$ wendet man \autoref{lem:claim23} auf $f - \sum_{m=0}^n h_m\colon A\to\RR$ an und erhält so
\[
h_{m+1}\colon X\to{} \left[-\tfrac{1}{3}\left(\tfrac{2}{3}\right)^{m+1}c, \tfrac{1}{3}\left(\tfrac{2}{3}\right)^{m+1}c\right].
\]
Dann konvergiert $\sum_{m=0}^\infty h_m$ gleichmáßig auf $X$ gegen eine stetige Funktion $F\colon X\to\RR$ mit $F(a) = f(a)$ für $a\in A$ und
\[
|F(x)| \leq \frac{c}{3}\sum_{m=0}^\infty \left(\tfrac{2}{3}\right)^{m} = \tfrac{c}{3}\cdot 3 = c.
\]

Angenommen, es ist sogar $|f(a)| < c$ für alle $a\in A$. In diesem Fall ist
\[
A_0 = \{x\in X : |F(x)| = c\}
\]
abgeschlossen und $A_0\cap A = \emptyset$. Es gibt also eine Urysohnfunktion $m\colon X\to{}[0,1]$ mit $m|_{A_0} = 0$ und $m|_{A} = 1$. Ersetzt man dann $F$ durch $F\cdot m$ erhält man eine Fortsetzung mit $|F(x)| < c$.

Ist $f\colon A\to\RR$ nicht beschränkt, sei $\widehat f(a) = \tfrac{f(a)}{1 + |f(a)|}$. Dann ist $\widehat f\colon A\to (-1,1)$ beschränkt und stetig. Es gibt also eine stetige Fortsetzung $\widehat g\colon X\to{}(-1,1)$ von $\widehat f$. Damit definiert $F(x) = \tfrac{\widehat g(x)}{1 - |\widehat g(x)|}$ eine stetige Funktion $F\colon X\to\RR$ mit $F(a) = f(a)$, \ddh~eine Fortsetzung von $f$.
\end{proof}

\begin{corollary}
Metrische Räume sind normal.\proofomitted
\end{corollary}

Natürlich ist nicht jeder Hausdorffraum regulär. Zum Beispiel kann man die folgende Basis für eine Topologie auf $\RR$ betrachten. Sie enthält zunächst alle offenen Intervalle der Form $(a,b)$ für $a,b\in\RR$, aber auch die Mengen $(a,b)\setminus\{\tfrac1n : n\in\ZZ_{>0}\}$. Wir schreiben vorübergehend $K = \{\tfrac1n : n\in\ZZ_{>0}\}$ und $\RR_K$ für $\RR$ mit der erzeugten Topologie. Die Topololgie auf $\RR_K$ ist feiner als die euklidische Topologie, also ist $\RR_K$ automatisch Hausdorff. Aber $\RR_K$ ist nicht regulär. Es ist nämlich $K\subset \RR_K$ abgeschlossen und $0\not\in K$. Gäbe es offene Mengen $U,V\subset\RR_K$ mit $0\in U$, $K\subset V$ und $U\cap V = \emptyset$, so könnten wir zuerst ohne Einschränkung annehmen, dass $U$ ein Basiselement für die Topologie auf $\RR_K$ ist. Aber $U=(a,b)$ ist unmöglich, weil $0\in(a,b)$ implizieren würde, dass ein $N\in\NN$ existiert mit $\tfrac1N\in (a,b)$. Also ist $U = (a,b)\setminus K$ mit $a<0<b$. Dann existiert aber wieder ein $n\in\NN$ mit $\tfrac1n\in(a,b)$. Aber es ist $\frac1n\in V$ und nach der definition der Topologie auf $\RR_K$ existiert dann ein offenes Intervall $(c,d)\subset V$ mit $\tfrac1n\in(c,d)$. Aber dann wäre $((a,b)\setminus K)\cap (c,d)\neq\emptyset$.

Wir hatten gesehen, dass Produkte von Hausdorffräumen wieder Hausdorffräume sind. Das gilt nicht für normale Räume. Zum Beispiel bilden die halboffenen Intervalle $[a,b)$ mit $a,b\in\RR$ eine Basis für eine Topologie auf $\RR$ und $\RR$ mit der erzeugten Topologie ist die \emph{Sorgenfreygerade} $\Rsf$. Der Raum $\Rsf$ ist normal, denn seien $A,B\subset\Rsf$ abgeschlossen und disjunkt. Für jedes $b\in B$ wähle ein $b'$ mit $[b,b')\subset \Rsf\setminus A$. Dann ist $V = \bigcup_{b\in B}[b,b')$ offen in $\Rsf$ und $B\subset V$. Analog wähle für jedes $a\in A$ ein $a''$ mit $[a,a'')\subset \Rsf\setminus B$. Dann ist auch $U = \bigcup_{a\in A} [a,a'')\subset\Rsf$ offen und $A\subset U$. Wäre nun aber $U\cap V\neq\emptyset$, dann gäbe es $a,b\in\RR$ mit $[a,a'')\cap [b,b')\neq\emptyset$ im Widerspruch zu $A\cap B = \emptyset$.

Wir betrachten nun die \emph{Sorgenfreyebene} $\Rsf^2 = \Rsf\times\Rsf$ mit der Produkttopologie. Dann ist $\Rsf^2$ nicht normal: Zuerst ist nämlich $\QQ^2\subset\Rsf^2$ dicht, denn für jede nichtleere offene Menge $U\subset \Rsf^2$ existiert ein halboffenes Rechteck $[a,b)\times[c,d)\subset U$ und $[a,b)\times[c,d)\cap\QQ^2\neq\emptyset$. Weiter ist die Antidiagonale $\Delta' = \{(x,-x): x\in\RR\}\subset\Rsf^2$ abgeschlossen, denn für $(x,y)\in\Rsf^2$ mit $x\neq -y$ gibt es zwei Fälle:
\begin{description}
\item[{$x > -y$:}] Dann ist $[x,x+1)\times[y,y+1)\subset \Rsf^2\setminus\Delta'$.
\item[{$x < -y$:}] Sei dann $\varepsilon = -(x+y)/2 > 0$. Dann ist $[x,x+\varepsilon)\times[y, y+\varepsilon)\subset \Rsf^2\setminus \Delta'$.
\end{description}
Außerdem ist $\Delta'$ sogar diskret in der Teilraumtopologie, denn für jedes $x\in\RR$ ist die Einpunktmenge $\{(x,-x)\} = \Delta'\cap [x,x+1)\times[-x,-x+1)$ offen in der Teilraumtopologie auf $\Delta'$. Nehmen wir jetzt an, $\Rsf^2$ wäre doch normal. Dann ist jede Teilmenge $A\subset\Delta'$ abgeschlossen in $\Rsf^2$ und es gäbe \emph{disjunkte} offene Mengen $U_A,V_A\subset\Rsf^2$ mit $A\subset U_A$ und $\Delta'\setminus A\subset V_A$. Definiere eine Abbildung
\[
\psi\colon\{A\subset\Delta' : A\neq\emptyset,\Delta'\}\to \powerset{\QQ},\quad A\mapsto U_A\cap \QQ^2.
\]
Diese Abbildung ist dann injektiv, denn seien $A,B\subset\Delta'$ mit $A\neq B$. Dann gibt es ein $z\in A$ mit $z\not\in B$ und dafür ist dann $z\in U_A$ und $z\in \Delta'\setminus B\subset V_B$. Aber dann ist $U_A\cap V_B$ nichtleer und offen in $\Rsf^2$. Also gibt es ein $q\in \QQ^2\cap U_A\cap V_B$ und für dieses $q$ ist dann $q\not\in U_B$. Also ist auch $\psi(A) = U_A\cap\QQ^2 \neq U_B\cap\QQ^2 = \psi(B)$. Aber $\psi$ kann aus Kardinalitätsüberlegungen nicht injektiv sein; die Kardinalität von $\powerset{Q}$ ist die Kardinalität des Kontinuums, aber die Kardinalität von $\powerset{\Delta'}\setminus\{\emptyset,\Delta'\}$ ist die der Potenzmenge des Kontinuums, also strikt größer.

Insbesondere haben wir gezeigt, dass $\Rsf^2$ nicht metrisierbar sein kann. Daraus folgt auch, dass $\Rsf$ selbst nicht metrisierbar ist, denn Produkte metrisierbarer Räume sind metrisierbar. Aber $\Rsf$ ist trotzdem normal.
\subsection{Zusammenhang}

\begin{definition}
Ein nichtleerer topologischer Raum $X$ ist \emph{zusammenhängend}, für offene Menge $U,V\subset X$ mit $U\cup V = X$ und $U\cap V = \emptyset$ bereits $U=\emptyset$ oder $V = \emptyset$. Alternativ aber äquivalent heißt ein Raum $X$ \emph{unzusammenhängend}, wenn es eine nichttriviale offene und abgeschlossene Menge in $X$ gibt, \ddh~ein $U\subset X$ mit $U\neq\emptyset$, $U\neq X$, aber $U$ offen und $U$ abgeschlossen.

Ein nichtleerer topologischer Raum $X$ ist \emph{wegzusammenhängend}, wenn es für alle $x,y\in X$ eine stetige Funktion $f\colon [0,1]\to X$ gibt mit $f(0) = x$ und $f(1) = y$. Eine solche Funktion heißt \emph{Weg} von $x$ nach $y$.
\end{definition}

\begin{theorem}[Zwischenwertsatz]\label{thm:intermediate-value}
Sei $X$ zusammenhängend und $f\colon X\to Y$ stetig. Dann ist $f(X)\subset X$ mit der Teilraumtopologie zusammenhängend.
\end{theorem}
\begin{proof}
Seien $U,V\subset Y$ offen mit $U\cap V\cap f(X) = \emptyset$ und $f(X) \subset U\cup V$. Dann ist $X = f^{-1}(U)\cup f^{-1}(V)$ mit den offenen Mengen $f^{-1}(U)$ und $f^{-1}(V)$. Außerdem ist $f^{-1}(U)\cap f^{-1}(V) = f^{-1}(U\cap V\cap f(X)) = \emptyset$. Also ist bereits $f^{-1}(U) = \emptyset$ oder $f^{-1}(V) = \emptyset$. Dann ist aber entsprechend $f(X)\cap U = \emptyset$ oder $f(X)\cap V = \emptyset$.
\end{proof}

\begin{theorem}\label{thm:closure-connected}
Sei $M\subset X$ zusammenhängend und $M\subset N\subset \cl{M}$. Dann ist auch $N$ zusammenhängend.
\end{theorem}
\begin{proof}
Seien $U,V\subset X$ offen mit $U\cap V\cap N = \emptyset$ und $N\subset U\cup V$. Aber dann ist auch $M\subset U\cup V$, also etwa ohne Einschränkung $U\cap M = \emptyset$. Also ist $M\subset X\setminus U$ und damit auch $N \subset \cl{M}\subset X\setminus U$, \ddh~$U\cap N = \emptyset$.
\end{proof}

\begin{lemma}
Sei $U\subset X$ eine Teilmenge und $\{u_n\}_{n\in\NN}$ eine Folge in $X$ mit $u_n\in U$ für alle $n\in\NN$. Angenommen $u_n\cto u$. Dann ist $u\in\cl{U}$.
\end{lemma}
\begin{proof}
Sei $A\supset U$ abgeschlossen in $X$. Wäre $u\not\in A$, so wäre $u\in X\setminus A$ und $X\setminus A$ ist offen. Also gäbe es insbesondere ein $n\in\NN$ mit $u_n\in X\setminus A$ im Widerspruch zu $u_n\in U$. Also ist $u\in A$ und da $A$ beliebig war schließen wir, dass $u\in\cl{U}$.
\end{proof}

\begin{theorem}
In $\RR$ sind Intervalle zusammenhängend.
\end{theorem}
\begin{proof}
Wegen \autoref{thm:closure-connected} können wir annehmen, dass das fragliche Intervall von der Form $(a,b)$ ist. Seien $U,V\subset (a,b)$ offen mit $U\cap V = \emptyset$ und $(a,b) = U\cup V$. Angenommen, es wäre $U\neq\emptyset$ und $V\neq\emptyset$. Setze $v = \sup V$. Dann gibt es eine Folge $(v_n)_{n\in\NN}$ mit $v_n\in V$ und $v_n\cto v$. Aber $V = (a,b)\setminus U$ ist abgeschlossen in $(a,b)$ und deshalb ist dann $v\in V$. Also ist $(v,b)\subset U$. Aber dann gibt es eine Folge $(u_n)_{n\in\NN}$ mit $u_n\in (v,b)\subset U$ und $u_n\cto v$. Aber $U = (a,b)\setminus V$ ist auch abgeschlossen in $(a,b)$, also wäre $v\in U$ im Widerspruch zu $U\cap V = \emptyset$.
\end{proof}

\begin{theorem}
Ein wegzusammenhängender Raum $X$ ist zusammenhängend.
\end{theorem}
\begin{proof}
Seien $U,V\subset X$ offen mit $U\cup V = X$ und $U\cap V = \emptyset$. Angenommen, es wären $U\neq\emptyset$ und $V\neq\emptyset$, etwa $u\in U$ und $v\in V$. Da $X$ wegzusammenhängend ist, gibt es eine stetige Funktion $f\colon [0,1]\to X$ mit $f(0) = u$ und $f(1) = v$. Dann ist nach \autoref{thm:intermediate-value} das Bild $M = f([0,1])$ zusammenhängend. Aber es wäre dann $(M\cap U)\cap (M\cap V) = \emptyset$, $(M\cap U)\cup (M\cap V) = M$, obwohl $u\in M\cap U$ und $v\in M\cap V$.
\end{proof}

Nicht jeder zusammenhängende Raum ist auch wegzusammenhängend. Sei zum Beispiel
\[
X = \{(x,\sin(1/x)) : x\in (0,1]\}\cup \{0\}\times[-1,1]
\]
die \emph{topologist's sine curve}. Dann ist $X$ zusammenhängend, denn für $\Gamma = \{(x,\sin(1/x)) : x\in (0,1]\}$ ist $X = \cl{\Gamma}$ und $\Gamma$ ist das Bild der zusammenhängenden Menge $(0,1]$ unter der stetigen Funktion $x\mapsto (x,\sin(1/x))$. Aber man überlegt sich, dass es keine stetige Funktion $f\colon[0,1]\to X$ mit $f(0) = (0,0)$ und $f(1) = (1,\sin(1))$.

\begin{definition}
Sei $X$ ein topologischer Raum. Wir definieren eine Áquivalenzrelation $\sim$ auf $X$, für die $x\sim y$ genau dann, wenn es einen Weg von $x$ nach $y$ in $X$ gibt. Der Quotient $\pi_0(X) = X/{\sim}$ ist die Menge der \emph{Wegzusammenhangskomponenten}.
\end{definition}

Die Anzahl der Wegzusammenhangskomponenten $\#\pi_0(X)$ eines Raums $X$ ist eine Homöomorphieinvariante. Ist nämlich $\varphi\colon X\to Y$ ein Homöomorphismus, so haben wir eine Abbildung $\varphi_*\colon \pi_0(X)\to\pi_0(Y)$ mit $\varphi([x]) = [\varphi(x)]$. Diese Abbildung ist tatsächlich wohldefiniert, denn für $x\sim y$ entlang eines Wegs $f\colon [0,1]\to X$ ist $\varphi\circ f\colon [0,1]\to Y$ ein Weg von $f(x)$ nach $f(y)$. Außerdem ist $\varphi_*$ bijektiv mit Umkehrabbildung $(\varphi^{-1})_*$. Insbesondere folgt, dass $\#\pi_0(X) = \#\pi_0(Y)$.

Zum Beispiel haben wir jetzt genügend technische Werkzeuge um zu sehen, dass es keinen Homöomorphismus $[0,2\pi)\to S^1$ geben kann. Man überlegt sich nämlich, dass $S^1\setminus \{x\}$ für jedes $x\in S^1$ wegzusammenhängend ist. Aber $[0,2\pi)\setminus\{\pi\} = [0,\pi)\cup (\pi,2\pi)$ ist unzusammenhängend. Gäbe es nun einen Homöomorphismus $\varphi\colon [0,2\pi)\to S^1$, dann wäre $\varphi|_{[0,2\pi)\setminus\{\pi\}}$ ein Homöomorphismus $[0,2\pi)\setminus\{\pi\}\to S^1\setminus\{\varphi(\pi)\}$. Aber $\#\pi_0(S^1\setminus\{\varphi(\pi)\}) = 1$ während $\#\pi_0([0,2\pi)\setminus\{\pi\}) = 2$.

\subsection{Kompakte Räume}

In $\RR^n$ heißt eine Teilmenge $K$ \emph{kompakt}, wenn $K$ abgeschlossen und beschränkt ist. Allgemeiner heißt in einem metrischen Raum eine Teilmenge $K$ kompakt, wenn jede Folge eine konvergente Teilfolge mit Grenzwert in $K$ zulässt. Beide Definitionen können nicht ohne Weiteres auf allgemeine topologische Räume angewandt werden, da einerseits ein Abstandsbegriff fehlt und andererseits Folgen im Allgemeinen die Topologie nicht bestimmen. Stattdessen führt man folgenden Begriff ein.

\begin{definition}
Ein topologischer Raum $X$ heißt \emph{kompakt}, falls jede offene Überdeckung von $X$ eine endliche Teilüberdeckung besitzt. Das heißt, gegeben eine Familie $\{U_i\}_{i\in I}$ von offenen Mengen $U_i\subset X$ mit $\bigcup_{i\in I} U_i = X$, existiert eine endliche Teilmenge $J\subset I$ mit $\bigcup_{i\in J} U_i = X$.

Eine Teilmenge $M\subset X$ heißt kompakt, wenn sie bezüglich der Teilraumtopologie ein kompakter Raum ist.
\end{definition}

Manche Autoren fordern, dass kompakte Räume Hausdorff sind. Wir verzichten aber darauf. Mit dieser Konvention sind zum Beispiel alle endlichen topologischen Räume kompakt. Unendliche Mengen $X$ mit der kofiniten Topologie sind auch kompakt. Ist nämlich $\{U_i\}_{i\in I}$ eine offene Überdeckung von $X$ und $i_0\in I$ mit $U_{i_0} \neq \emptyset$. Dann ist $X\setminus U_{i_0}$ endlich und $\{U_i\setminus U_{i_0}\}_{i\neq i_0}$ eine offene Überdeckung von $X\setminus U_{i_0}$. Aber $X \setminus U_{i_0}$ ist kompakt, also existiert eine endliche Teilmenge $J\subset I$ mit
\[
X\setminus U_{i_0} = \bigcup_{\substack{i\in J\\ i\neq i_0}} U_i\setminus U_{i_0}.
\]
Aber dann ist $X = \bigcup_{i\in J} U_i \cup U_{i_0}$.

Die reellen Zahlen $\RR$ bilden mit der euklidischen Topologie keinen kompakten Raum. Etwa besitzt die Überdeckung
\[
\RR = \cdots \cup (-2,0) \cup (-1,1) \cup (0,2) \cup \cdots
\]
keine endliche Teilüberdeckung.

\begin{lemma}\label{lem1}
Sei $X$ ein topologischer Raum.
\begin{enumerate}
\item Ist $X$ kompakt, so ist jede abgeschlossene Teilmenge $A\subset X$ kompakt.
\item Ist $X$ Hausdorff, so ist jede kompakte Teilmenge $K\subset X$ abgeschlossen.
\end{enumerate}
\end{lemma}
\begin{proof}\needspace{2\baselineskip}\leavevmode
\begin{enumerate}
\item Sei $A\subset X$ abgeschlossen und $\{U_i\}_{i\in I}$ eine offene Überdeckung von $A$. Dann existieren offene Teilmengen $V_i\subset X$ mit $U_i = V_i\cap A$ und $X = \bigcup_{i\in I} V_i \cup X\setminus A$ ist eine offene Überdeckung von $X$. Folglich existiert eine endliche Teilmenge $J\subset I$ mit $X = \bigcup_{i\in J} V_i \cup X\setminus A$. Dann ist aber auch $A = X\cap A = \bigcup_{i\in J} U_i$.
\item Sei $K\subset X$ kompakt und $x\in X\setminus K$. Dann existiert eine offene Menge $U\subset X\setminus K$ mit $U\cap K = \emptyset$: Da $X$ Hausdorff ist, existiert nämlich für jedes $y\in K$ eine offene Umgebung $U_y$ von $x$ und eine offene Umgebung $V_y$ von $y$ mit $U_y\cap V_y=\emptyset$. Dann ist $K \subset \bigcup_{y\in K} V_y$ eine offene Überdeckung von $K$ und folglich existert eine endliche Teilmenge $K_0\subset K$ mit $K\subset \bigcup_{y\in K_0} V_y$. Dann ist aber $\bigcap_{y\in K_0} U_y \subset X\setminus K$ eine offene Umgebung von $x$. Das genügt um zu sehen, dass $X\setminus K$ offen ist. \qedhere
\end{enumerate}
\end{proof}

\begin{definition}
Eine Abbildung $f\colon X\to Y$ zwischen topologischen Räumen $X$ und $Y$ heißt \emph{abgeschlossen}, falls für alle abgeschlossenen Teilmengen $A\subset X$ das Bild $f(A)\subset Y$ abgeschlossen ist.
\end{definition}

Ähnlich wie für offene Abbildungen ist eine stetige Bijektion ist genau dann ein Homöomorphismus, wenn sie eine abgeschlossene Abbildung ist.

\begin{lemma}\label{lem2}
Seien $X$ und $Y$ topologische Räume, $X$ kompakt und $f\colon X\to Y$ stetig.
\begin{enumerate}
\item Ist $f$ surjektiv, so ist $Y$ kompakt.
\item Ist $Y$ Hausdorff, so ist $f$ abgeschlossen.
\item Ist $Y$ Hausdorff und $f$ bijektiv, so ist $f$ ein Homöomorphismus.
\end{enumerate}
\end{lemma}
\begin{proof}
Natürlich implizieren (i) und (ii) zusammen mit der vorstehenden Bemerkung Teil (iii).
\begin{enumerate}
\item Sei $\{U_i\}_{i\in I}$ eine offene Überdeckung von $Y$. Dann ist $\{f^{-1}(U_i)\}_{i\in I}$ eine offene Überdeckung von $X$. Also existiert eine endliche Teilmenge $J\subset I$ mit $\bigcup_{i\in J} f^{-1}(U_i) = X$. Da $f$ surjektiv ist, gilt dann auch $Y = \bigcup_{i\in J} U_i$.
\item Sei $A\subset X$ abeschlossen. Dann ist $A$ kompakt nach \autoref{lem1} und damit nach (i) auch das Bild $f(A)\subset Y$ kompakt. Aber da $Y$ Hausdorff ist, folgt nach \autoref{lem1}, dass $f(A)$ abgeschlossen ist.\qedhere
\end{enumerate}
\end{proof}

Im Allgemeinen sind Teilräume kompakter Räume nicht selbst kompakt. Zum Beispiel ist keine Teilmenge eines kompakten Hausdorffraums, die nicht abgeschlossen ist, kompakt. Hingegen zeigt \autoref{lem2}, dass Quotienten kompakter Räume immer selbst kompakt sind.

\begin{corollary}[Heine-Borel]\needspace{2\baselineskip}\leavevmode
\begin{enumerate}
\item Das Intervall $[0,1]$ ist kompakt.
\item Eine Teilmenge von $\RR^m$ ist kompakt, genau dann, wenn sie abgeschlossen und beschränkt ist.
\end{enumerate}
\end{corollary}
\begin{proof}\needspace{2\baselineskip}\leavevmode
\begin{enumerate}
\item Sei $X = \{0,1,2,\dots,9\}$ mit der diskreten Topologie. Dann ist $X$ kompakt und wegen \autoref{thm:tychonoff} auch das Produkt $X^\NN = \prod_{n\in\NN} X$ kompakt. Definiere eine Abbildung
\[
f\colon X^\NN \to{} [0,1],\quad f(a_1,a_2,\dots) = \sum_{n\in \NN} \frac{a_n}{10^n} = 0.a_1a_2\dots
\]
Dann ist $f$ surjektiv und stetig(!). Wegen \autoref{lem2} ist dann $f(X^\NN) = [0,1]$ kompakt.
\item Sei zuerst $A\subset\RR^m$ abgeschlossen und beschränkt. Insbesondere gibt es dann ein $n\in\NN$, so dass $A\subset [-n,n]^m$. Aber $[0,1]\isom [-n,n]$ ist kompakt und damit ist nach \autoref{thm:tychonoff} auch $[-n,n]^m$ kompakt. Nach \autoref{lem1} ist dann auch $A$ als abgeschlossene Teilmenge eines kompakten Raums selbst kompakt.
\item Sei $K\subset \RR^m$ kompakt. Dann ist $K$ nach \autoref{lem1} abgeschlossen. Außerdem ist $K\subset \bigcup_{n\in\NN} B_n(0) = \RR^m$ eine offene Überdeckung. Folglich gibt es eine endliche Teilüberdeckung, \ddh~ein $N\in\NN$ mit $K\subset \bigcup_{n=1}^N B_n(0) = B_N(0)$. Anders gesagt muss $K$ beschränkt sein.
\end{enumerate}
\end{proof}

\begin{definition}
Sei $X$ ein topologischer Raum und $A\subset X$ eine beliebige Teilmenge. Ein \emph{Häufungspunkt} von $A$ ist ein $p\in X$, so dass jede Umgebung von $p$ einen Punkt $p'\in A$ mit $p'\neq p$ enthält.
\end{definition}
\begin{theorem}\label{thm:limit-points-compact}
Ist $X$ kompakt, so besitzt jede unendliche Teilmenge $A\subset X$ einen Häufungspunkt.
\end{theorem}
\begin{proof}
Andernfalls ist zuerst $A$ abgeschlossen: Wäre nämlich $p\in\cl{A}\setminus A$, so wäre für jede Umgebung $U$ von $p$ der Schnitt $U\cap A\neq \emptyset$. Aber dann wäre $p$ ein Häufungspunkt von $A$.

Außerdem gibt es, falls $A$ keinen Häufungspunkt besitzt, für alle $p\in A$ eine Umgebung $U_p$ von $p$ mit $A\cap U_p = \{p\}$. Folglich ist die Teilraumtopologie auf $A$ diskret. Aber da $X$ kompakt ist, ist nach \autoref{lem1} auch $A$ kompakt. Aber jeder diskrete, kompakte Raum ist endlich.
\end{proof}

\begin{definition}
Ein topologischer Raum $X$ heißt \emph{folgenkompakt}, wenn jede Folge in $X$ eine konvergente Teilfolge besitzt.
\end{definition}
Im Allgemeinen sind Folgenkompaktheit und Kompaktheit unabhängig voneinander. Es gibt folgenkompakte Räume, die nicht kompakt sind, und kompakte Räume, die nicht folgenkompakt sind.

\begin{theorem}
Sei $(X,d)$ ein metrischer Raum. Dann sind die folgenden Aussagen äquivalent:
\begin{enumerate}
\item $X$ ist kompakt.
\item Jede unendliche Teilmenge von $X$ hat einen Häufungspunkt.
\item $X$ ist folgenkompakt.
\end{enumerate}
\end{theorem}
\begin{proof}\needspace{2\baselineskip}\leavevmode
\begin{enumerate}
\item[(1\cto2)] Das ist \autoref{thm:limit-points-compact}.
\item[(2\cto3)] Sei $\{x_n\}_{n\in\NN}$ eine Folge in $X$ und $A = \{x_n : n\in\NN \}$. Ist $A$ endlich, gibt es eine konstante und damit konvergente Teilfolge. Ist $A$ unendlich, besitzt $A$ einen Häufungspunkt $x\in X$. Dann existiert für jedes $m\in\NN$ ein $x_{n_m}\in A\cap B_{1/m}(x)$ mit $x_{n_m}\neq x$ und $n_m < n_{m+1}$. Dann konvergiert $\{x_{n_m}\}_{m\in\NN}$ gegen $x$.
\item[(3\cto1)] Sei $X = \bigcup_{i\in I} U_i$ eine offene Überdeckung von $X$. Für eine beliebige Teilmenge $B\subset X$ schreiben wir
\[
\diam(B) = \sup\{ d(x,y) : x,y\in B \}.
\]
Dann gibt es ein $\delta >0$, so dass für alle $B\subset X$ mit $\diam(B) < \delta$ ein $i\in I$ existiert mit $B\subset U_i$. Andernfalls gäbe es für alle $n\in\NN$ eine Teilmenge $\emptyset\neq C_n\subset X$ mit $\diam(C_n) < \tfrac{1}{n}$ und $C_n\not\subset U_i$ für alle $i\in I$. Sei $x_n\in C_n$ und $\{x_{n_m}\}_{m\in\NN}$ eine konvergente Teilfolge von $\{x_n\}_{n\in\NN}$, etwa $x_{n_m}\cto x$. Sei $j\in I$ mit $U = U_j\ni x$. Dann gibt es ein $\varepsilon>0$ mit $B_\varepsilon(x)\subset U$. Für $k\gg 0$ gilt dann $\tfrac{1}{n_k} < \tfrac{\varepsilon}{2}$ und $d(x_{n_k},x) < \tfrac{\varepsilon}{2}$. Aber dann wäre $C_{n_k}\subset B_{\varepsilon/2}(x_{n_k})\subset B_{\varepsilon}(x)\subset U$.

Weiter gibt es für alle $\varepsilon>0$ Punkte $x_1,\dots,x_m\in X$, so dass $X = B_\varepsilon(x_1)\cup\dots\cup B_\varepsilon(x_m)$. Andernfalls könnten wir induktiv eine Folge $\{x_n\}_{n\in\NN}$ in $X$ konstruieren. Wähle $x_1\in X$ beliebig. Für $n>1$ ist
\[
B = B_\varepsilon(x_1)\cup\dots\cup B_\varepsilon(x_{n-1})\neq X.
\]
Wähle $x_n\in X\setminus B$. Aber dann kann $\{x_n\}_{n\in\NN}$ keine konvergente Teilfolge besitzen.

Für $\delta$ wie oben sei $\varepsilon = \tfrac{\delta}{3}$. Dann gibt es $x_1,\dots,x_m\in X$ mit $X = B_\varepsilon(x_1)\cup\dots\cup B_\varepsilon(x_m)$. Dann ist $\diam(B_\varepsilon(x_i))\leq 2\varepsilon < \delta$ und dami existiert für jedes $i\in\{1,\dots,m\}$ ein $j_i\in I$ mit $B_\varepsilon(x_{j_i})\subset U_{j_i}$. Aber dann ist
\[
X = U_{j_1}\cup\dots\cup U_{j_m}.\qedhere
\]
\end{enumerate}
\end{proof}

\begin{definition}
Eine Teilmenge $A$ in einem metrischen Raum $(X,d)$ ist \emph{totalbeschränkt}, wenn es zu jedem $\varepsilon>0$ eine endliche Menge von Punkten $x_1,\dots,x_n\in A$ existiert, so dass $A\subset \bigcup_{i=1}^n B_\varepsilon(x_i)$.
\end{definition}

\begin{definition}
Eine Folge $\{x_n\}_{n\in\NN}$ in einem metrischen Raum $(X,d)$ ist eine \emph{Cauchyfolge}, falls für jedes $\varepsilon>0$ ein $N\in\NN$ existiert, so dass $d(x_m,x_n)<\varepsilon$ für $m,n\geq N$.

Der metrische Raum $(X,d)$ heißt \emph{vollständig}, wenn jede Cauchyfolge in $X$ konvergiert.
\end{definition}

Eine totalbeschränkte Teilmenge eines metrischen Raums ist natürlich auch beschränkt. Die Umkehrung gilt aber nicht. Jede kompakte Teilmenge $A$ eines metrischen Raums $(X,d)$ ist totalbeschränkt: Für $\varepsilon>0$ hat die offene Überdeckung $A\subset \bigcup_{x\in A} B_\varepsilon(x)$ nämlich eine endliche Teilüberdeckung.

Jede kompakte Teilmenge $A$ ist auch vollständig. Ist nämlich $\{x_n\}_{n\in\NN}$ eine Cauchyfolge in $A$, so existiert eine konvergente Teilfolge $\{x_{n_k}\}_{k\in\NN}$; etwa $x_{n_k}\cto x\in A$. Dann konvergiert auch $\{x_n\}$ gegen $x$, denn sei $\varepsilon>0$ und $N\in\NN$ so groß, dass $d(x_n, x_{n_N}) < \varepsilon/2$ für alle $n\geq N$ und $d(x, x_N) < \varepsilon/2$.  Dann ist $d(x,x_n) \leq d(x,x_{n_N}) + d(x_{n_N}, x_n) < \varepsilon$ für $n\geq N$. Ohne Beweis erwähnen wir das folgende Resultat.

\begin{theorem}
Ein totalbeschränkter und vollständiger metrischer Raum ist kompakt.\proofomitted
\end{theorem}

\subsection{Ultrafilter und der Satz von Tychonoff}
TODO

\begin{definition}
Für $f\colon X\to Y$ eine Abbildung zwischen zwei Mengen $X$ und $Y$ und $\cal F$ ein Filter auf $X$, definieren wir
\[
f_*\cal F \coloneqq \{M\subset Y : f^{-1}(M)\in\cal F\},
\]
den \emph{Pushforward-Filter}.
\end{definition}

\begin{lemma}
Sei $f\colon X\to Y$ eine Abbildung zwischen zwei Mengen $X$ und $Y$ und $\cal F$ ein Filter auf $X$.
\begin{enumerate}
\item Der Pushforward-Filter $f_*\cal F$ ist tatsächlich ein Filter auf $Y$.
\item Wenn $\cal F$ ein Ultrafilter ist, so ist auch $f_*\cal F$ ein Filter.
\end{enumerate}
\end{lemma}
\begin{proof}\needspace{2\baselineskip}\leavevmode
\begin{enumerate}
\item Wir haben $Y\in f_*\cal F$, denn $f^{-1}(Y) = X\in\cal F$, und $\emptyset\not\in f_*\cal F$, denn $f^{-1}(\emptyset) = \emptyset\not\in\cal F$. Für $M,N\in f_*\cal F$ ist $f^{-1}(M\cap N) = f^{-1}(M)\cap f^{-1}(N)\in\cal F$, also auch $M\cap N\in f_*\cal F$.

Ist weiter $M\in f_*\cal F$ und $N\supset M$, so ist $f^{-1}(M)\in\cal F$ und wegen $f^{-1}(N)\supset f^{-1}(M)$ auch $f^{-1}(N)\in \cal F$. Also ist auch $N\in f_*\cal F$.
\item Sei $M\subset Y$. Dann ist entweder $f^{-1}(M)\in\cal F$ oder $X\setminus f^{-1}(M) = f^{-1}(Y\setminus M)\in\cal F$. Aber das heißt, dass entweder $M\in f_*\cal F$ oder $Y\setminus M\in f_*\cal F$.\qedhere
\end{enumerate}
\end{proof}

\begin{theorem}[Tychonoff]\label{thm:tychonoff}
Beliebige Produkte kompakter Räume sind kompakt in der Produkttopologie.
\end{theorem}
\begin{proof}
Sei $I$ beliebige Indexmenge und $\{X_i\}_{i\in I}$ eine Familie kompakter topologischer Räume. Sei $X = \prod_{i\in I} X_i$ nichtleer und $\pi_j\colon X\to X_j$ die Projektion auf $X_j$ für jedes $j\in I$. Wir zeigen, dass $X$ kompakt ist, indem wir zeigen, dass jeder Ultrafilter in $X$ konvergiert.

Sei also $\cal F$ ein Ultrafilter auf $X$. Dann ist für jedes $i\in I$ der Pushforward-Filter $(\pi_i)_*\cal F$ ein Ultrafilter auf $X_i$. Weil $X_i$ kompakt ist, konvergiert $(\pi_i)_*\cal F$ gegen ein $x_i\in X_i$. Wir zeigen, dass dann $(x_i)_{i\in I}$ ein Grenzwert von $\cal F$ ist. Sei dafür $U\subset X$ eine offene Umgebung von $x$. Wegen dem dritten Filteraxiom können wir ohne Einschränkung $U$ durch eine offene Menge der Form $\prod_{i\in I}\pi_i^{-1}(U_i)$ für offene Mengen $U_i\subset X_i$ mit $x_i\in U_i$ ersetzen, wobei aber nur endlich viele der $U_i$ verschieden von $X_i$ sind. Sei etwa $J\subset I$ eine endliche Teilmenge mit $U_i = X_i$ für $i\not\in J$.

Aber dann ist $U_i$ für jedes für jedes $i\in I$ eine offene Umgebung von $x_i$. Da $(\pi_i)_*\cal F$ gegen $x_i$ konvergiert, ist dann $U_i\in (\pi_i)_*\cal F$, \ddh~$\pi_i^{-1}(U_i)\in\cal F$. Aber dann ist $\bigcap_{i\in J} \pi_i^{-1}(U_i)\in\cal F$, da $J$ endlich ist. Außerdem ist
\[
\prod_{i\in I} U_i = \bigcap_{i\in J}\pi_i^{-1}(U_i).
\]
Also konvergiert $\cal F$ tatsächlich gegen $(x_i)_{i\in I}$.
\end{proof}

Der Beweis funktioniert nicht, wenn man die Produkttopologie durch die Boxtopologie ersetzt. Ist zum Beispiel $X$ ein endlicher diskreter topologischer Raum, so ist $\prod_{n\in\NN} X$ mit der Produkttopologie wie eben gezeigt kompakt. Aber die Boxtopologie auf $\prod_{n\in\NN} X$ ist diskret!

\subsection{Einpunktkompaktifizierung}
Das offene Intervall $(0,1)$ mit der euklidischen Topologie ist nicht kompakt. Aber es gibt eine stetige Funktion $(0,1)\to S^1$ deren Bild nur einen Punkt in $S^1$ nicht enthält und $S^1$ ist kompakt. Das ist ein erstes Beispiel für eine Einpunktkompaktifizierung.

\begin{definition}
Sei $X$ ein Hausdorffraum und $X^* = X\cup\{\infty\}$, wobei $\infty\not\in X$. Sei $\cal T_\infty$ die folgende Menge von Teilmengen von $X^*$:
\begin{enumerate}
\item Für jede offene Teilmenge $U\subset X$ ist $U\in\cal T_\infty$.
\item Für $K\subset X$ kompakt ist $X^*\setminus K\in\cal T_\infty$.
\end{enumerate}
Der topologische Raum $(X,\cal T_\infty)$ heißt \emph{Einpunktkompaktifizierung} von $X$.
\end{definition}

\begin{lemma}
Die Familie $\cal T_\infty$ ist eine Topologie auf $X^*$.
\end{lemma}
\begin{proof}
Wir haben $\emptyset\in\cal T_\infty$, da $\emptyset\subset X$ offen ist. Außerdem ist $X^*\in\cal T_\infty$, denn $X^*\setminus \emptyset = X^*$ und $\emptyset\subset X$ ist kompakt.

Für $U_1,U_2\subset X$ offen ist natürlich auch $U_1\cap U_2\in\cal T_\infty$. Ist $U\subset X$ offen und $V = X^*\setminus K$ mit $K\subset X$ kompakt, so ist $U\cap V = U\cap (X^*\setminus K) = U\cap (X\setminus K)\subset X$ offen in $X$, denn $K$ ist als kompakte Menge in einem Hausdorffraum abgeschlossen. Sind weiter $K_1,K_2\subset X$ kompakt, so ist $(X^*\setminus K_1)\cap (X^*\setminus K_2) = X^*\setminus (K_1\cup K_2)\in\cal T_\infty$, da $K_1\cup K_2\subset X$ kompakt ist.

Ist $\{U_i\}_{i\in I}$ eine Familie von offenen Teilmengen von $X$, so ist natürlich auch $\bigcup_{i\in I} U_i\in\cal T_\infty$. Ist $\{K_i\}_{i\in I}$ eine Familie kompakter Teilmengen von $X$, so ist
\[
\bigcup_{i\in I} (X^*\setminus K_i) = X^*\setminus \bigcap_{i\in I} K_i \in \cal T_\infty,
\]
da $\bigcap_{i\in I} K_i\subset X$ kompakt ist.

Es bleibt zu zeigen, dass $U\cup (X^*\setminus K)\in\cal T_\infty$ für $U\subset X$ offen und $K\subset X$ kompakt. Dann ist $U\cup (X^*\setminus K) = X^*\setminus (K\setminus K\cap U)$ und $K\setminus (K\cap U) = K\cap (X\setminus U)\subset K$ ist eine abgeschlossene Teilmenge der kompakten Menge $K$. Also ist auch $U\cup (X^*\setminus K)\in\cal T_\infty$.
\end{proof}
\begin{lemma}
Die Teilraumtopologie auf $X\subset X^*$ stimmt der ursprünglichen Topologie auf $X$ überein.
\end{lemma}
\begin{proof}
Ist $U\subset X$ offen in der ursprünglichen Topologie auf $X$, so ist auch $U\subset X^*$ offen und $U = X\cap U$.

Sei $U\subset X^*$ eine beliebige offene Teilmenge. Ist $U\subset X$, so ist $X\cap U = U$ offen in der ursprünglichen Topologie auf $X$. Ist $U = X^*\setminus K$ für $K\subset X$ kompakt, so ist $X\cap U = X\setminus K$ auch offen in der ursprünglichen Topologie auf $X$, denn $X$ ist Hausdorff und $K$ ist als kompakte Menge in einem Hausdorffraum abgeschlossen.
\end{proof}

\begin{lemma}
Der Raum $X^*$ ist kompakt.
\end{lemma}
\begin{proof}
Sei $X^* = \bigcup_{i\in I} U_i$ mit $U_i\subset X^*$ offen. Dann gibt es ein $i_0\in I$ mit $\infty\in U_{i_0}$ und dafür ist $U_{i_0} = X^*\setminus K$ für $K\subset X$ eine kompakte Teilmenge. Aber dann ist $K\subset \bigcup_{i\in I} X\cap U_i$ und es gibt eine endliche Teilmenge $J\subset I$ mit $K\subset \bigcup_{i\in J} X\cap U_i$. Aber dann ist $X^* = U_{i_0}\cup \bigcup_{i\in J} U_i$ eine endliche Teilüberdeckung.
\end{proof}

\begin{lemma}
Sei $X$ \emph{lokalkompakt}, \ddh~$X$ ist Hausdorff und für jedes $x\in X$ gibt es eine offene Umgebung $U\subset X$ von $x$ und eine kompakte Menge $K\subset X$ mit $x\in U\subset K$. Dann ist $X^*$ ein Hausdorffraum.
\end{lemma}
\begin{proof}
Seien $x\neq y$ Punkte in $X^*$. Sind $x,y\in X$, so gibt es offene Mengen $U,V\subset X$ mit $x\in U$, $y\in V$ und $U\cap V = \emptyset$, da wir annehmen, dass $X$ Hausdorff ist. Sei also einer der Punkte $\infty\in X^*$, etwa $x\in X$ und $y=\infty$. Da $X$ lokalkompakt ist, gibt es eine offene Umgebung $U\subset X$ von $x$ und eine kompakte Teilmenge $K\subset X$ mit $x\in U\subset K$. Aber dann ist $\infty\in X^*\setminus K$ und $X^*\setminus K\subset X^*$ ist offen. Außerdem ist $U\cap(X^*\setminus K) = \emptyset$.
\end{proof}

Insgesamt haben wir folgenden Satz gezeigt.
\begin{theorem}
Sei $X$ ein Hausdorffraum und $X^* = X\cup\{\infty\}$ die Einpunktkompaktifizierung von $X$. Dann gilt:
\begin{enumerate}
\item Die Teilraumtopologie auf $X\subset X^*$ stimmt mit der ursprünglichen Topologie auf $X$ überein.
\item $X^*$ ist kompakt.
\item Wenn $X$ lokalkompakt ist, dann ist $X^*$ Hausdorff.
\end{enumerate}
\end{theorem}

\begin{lemma}\label{lem:loccomp-compactification-hausdorff}
Sei $Y$ ein kompakter Hausdorffraum und $X\subset Y$ ein Teilraum, so dass $Y\setminus X = \{\pt\}$ nur einen Punkt enthält. Dann ist $X$ lokalkompakt.
\end{lemma}
\begin{proof}
Sei $x\in X$. Da $Y$ Hausdorff ist, gibt es offene Teilmengen $U,V\subset Y$ mit $x\in U$, $\pt\in V$ und $U\cap V = \emptyset$. Aber dann ist $Y\setminus V$ abgeschlossen und als Teilmenge des kompakten Raums $Y$ selbst kompakt. Außerdem ist $x\in X\cap U\subset Y\setminus V$.
\end{proof}

\begin{lemma}
Sei $X$ lokalkompakt und $X^*$ die Einpunktkompaktifizierung. Angenommen, $Y$ ist ein kompakter Hausdorffraum, so dass $X\subset Y$ ein Teilraum und $Y\setminus X = \{\pt\}$ ist. Dann ist die Abbildung $\varphi\colon Y\to X^*$ mit $\varphi|_X = \id_X$ und $\varphi(\pt) = \infty$ ein Homöomorphismus.
\end{lemma}
\begin{proof}
Die Abbildung $\varphi$ ist offenbar eine Bijektion. Da $Y$ kompakt und $X^*$ nach \autoref{lem:loccomp-compactification-hausdorff} Hausdorff ist, genügt es also zu zeigen, dass $\varphi$ stetig ist. Sei dafür $U\subset X^*$ offen.

Ist $U\subset X$, so ist $\varphi^{-1}(U) = U\subset X\subset Y$ offen in der Teilraumtopologie auf $X$. Aber $X\subset Y$ ist offen, da $\{\pt\}\subset Y$ abgeschlossen ist, also ist $U$ auch offen in $Y$. Ist hingegen $U = X^*\setminus K$ für eine kompakte Teilmenge $K\subset X$, so ist $\varphi^{-1}(U) = Y\setminus \varphi^{-1}(K) = Y\setminus K$. Aber $K\subset Y$ ist kompakt (in der Teilraumtopologie) und damit abgeschlossen, da $Y$ ein Hausdorffraum ist. Also ist auch dann $\varphi^{-1}(U)$ offen.
\end{proof}

Mit diesem Lemma können wir auch sehen, dass $S^1\isom (0,1)^*$. Es ist nämlich leicht zu sehen, dass $S^1\setminus\{\pt\}\isom (0,1)$ für jeden Punkt $\pt\in S^1$. Außerdem ist $S^1$ ein kompakter Hausdorffraum und $(0,1)$ ist lokalkompakt. Ähnlich ist die Einpunktkompaktifizierung von $\RR^n$ homöomorph zur $n$-Sphäre $S^n$.

\section{Algebraische Topologie}
TODO

\begin{definition}
Sei $X$ ein topologischer Raum und $A\subset X$. Eine \emph{Homotopie relativ $A$} von $f\colon X\to Y$ nach $g\colon X\to Y$ ist eine Homotopie $H\colon X\times[0,1]\to Y$ von $f$ nach $g$, so dass $H(a,t) = f(a) = g(a)$ für alle $a\in A$. Wir schreiben $f\simeq_A g$.
\end{definition}

Wieder ist $\simeq_A$ eine Äquivalenzrelation. Der Beweis ist komplett analog zu dem für $\simeq$. Das wichtigste Beispiel für eine relative Homotopie findet sich bei Wegen. Zwei Wege $\gamma_1\colon [0,1]\to X$ und $\gamma_2\colon[0,1]\to X$ heißen nämlich \emph{homotop}, $\gamma_1\simeq\gamma_2$, wenn sie homotop relativ $\{0,1\}$ sind. Wir bezeichnen mit $[\gamma]$ die Homotopieklasse (relativ $\{0,1\}$) von einem Weg $\gamma$. Um die Notation etwas zu vereinfachen schreiben wir oft $I\coloneqq [0,1]$ für das Einheitsintervall.

Seien zum Beispiel $X = \RR^n$ mit der euklidischen Topologie und $\gamma_1,\gamma_2\colon I \to \RR^n$ zwei Wege mit $\gamma_1(0) = \gamma_2(0)$ und $\gamma_1(1) = \gamma_2(1)$. Dann ist immer $\gamma_1\simeq\gamma_2$, denn
\[
I\times I\to \RR^n,\quad (s,t)\mapsto (1-t)\gamma_1(s) + t\gamma_2(s)
\]
ist eine Homotopie relativ $\{0,1\}$.

\begin{definition}
Sei $X$ ein topologischer Raum mit Wegen $\gamma_1,\gamma_2\colon I\to X$, so dass $\gamma_1(1) = \gamma_2(0)$. Dann definieren wir einen Weg $\gamma_1*\gamma_2\colon I\to X$, die \emph{Verknüpfung} von $\gamma_1$ und $\gamma_2$, durch
\[
(\gamma_1*\gamma_2)(s) = \begin{cases}
\gamma_1(2s) & s\in[0,\tfrac12]\\
\gamma_2(2s-1) & s\in[\tfrac12, 1].
\end{cases}
\]
\end{definition}

\begin{lemma}
Sei $X$ ein topologischer Raum mit Wegen $\gamma_1,\gamma_2\colon I\to X$, so dass $\gamma_1(1) = \gamma_2(0)$. Seien weiter $\gamma_1',\gamma_2'\colon I\to X$ Wege mit $\gamma_1\simeq_{\{0,1\}}\gamma_1'$ und $\gamma_2\simeq_{\{0,1\}}\gamma_2'$. Dann ist $\gamma_1*\gamma_2\simeq_{\{0,1\}} \gamma_1'*\gamma_2'$. Insbesondere ist die Verknüpfung
\[
([\gamma_1],[\gamma_2]) \mapsto [\gamma_1]\cdot[\gamma_2] \coloneqq [\gamma_1*\gamma_2]
\]
von Homotopieklassen wohldefiniert.
\end{lemma}
\begin{proof}
Sei $H\colon I\times I\to X$ eine Homotopie von $\gamma_1$ nach $\gamma_1'$. Wir definieren
\[
H'\colon I\times I\to X,\quad (s,t)\mapsto\begin{cases}
H(2s, t) & s\in[0,\tfrac12] \\
\gamma_2(2s-1) & s\in[\tfrac12, 1].
\end{cases}
\]
Dann ist $H'$ eine Homotopie von $\gamma_1*\gamma_2$ nach $\gamma_1'*\gamma_2$, denn $H'$ ist stetig und es gilt
\begin{align*}
H'(s,0) &= \begin{cases}
\gamma_1(2s) = H(2s,0) & s\in[0,\tfrac12] \\
\gamma_2(2s-1) & s\in[\tfrac12,1]
\end{cases} \\
\shortintertext{und}
H'(s,1) &= \begin{cases}
\gamma_1'(2s) = H(2s,1) & s\in[0,\tfrac12] \\
\gamma_2(2s-1) & s\in[\tfrac12,1].
\end{cases}
\end{align*}
Sei weiter $G\colon I\times I\to X$ eine Homotopie von $\gamma_2$ nach $\gamma_2'$. Wir definieren
\[
G'\colon I\times I\to X,\quad (s,t)\mapsto\begin{cases}
\gamma_1'(2s) & s\in[0,\tfrac12] \\
G(2s-1, t) & s\in[\tfrac12, 1].
\end{cases}
\]
Wieder ist $G'$ eine Homotopie von $\gamma_1'*\gamma_2$ nach $\gamma_1'*\gamma_2'$. Also ist insgesamt $\gamma_1*\gamma_2\simeq\gamma_1'*\gamma_2\simeq\gamma_1'*\gamma_2'$.
\end{proof}

\begin{lemma}\label{lem:reparam}
Sei $\gamma\colon I\to X$ eine Weg in einem topologischen Raum $X$ und $\varphi\colon I\to I$ eine stetige Funktion mit $\varphi(0) = 0$ und $\varphi(1) = 1$. Dann ist $\gamma\circ\varphi\simeq_{\{0,1\}}\gamma$.
\end{lemma}
\begin{proof}
Die Abbildung
\[
H\colon I\times I\to X,\quad (s,t)\mapsto \gamma(t\varphi(s) + (1-t)s)
\]
ist eine Homotopie relativ $\{0,1\}$ von $\gamma$ nach $\gamma\circ\varphi$.
\end{proof}

\begin{lemma}\label{lem:path-assoc}
Sei $X$ ein topologischer Raum mit Wegen $\gamma_1,\gamma_2,\gamma_3\colon I\to X$, so dass $\gamma_1(1) = \gamma_2(0)$ und $\gamma_2(1) = \gamma_3(0)$. Dann ist $(\gamma_1*\gamma_2)*\gamma_3 \simeq_{\{0,1\}} \gamma_1*(\gamma_2*\gamma_3)$, also $([\gamma_1]\cdot[\gamma_2])\cdot[\gamma_3] = [\gamma_1]\cdot([\gamma_2]\cdot[\gamma_3])$.
\end{lemma}
\begin{proof}
Wir haben $((\gamma_1*\gamma_2)*\gamma_3)\circ\varphi = \gamma_1*(\gamma_2*\gamma_3)$ für
\[
\varphi\colon I\to I,\quad s\mapsto\begin{cases}
\tfrac12 s & s\in[0,\tfrac12] \\
s - \tfrac14 & s\in[\tfrac12,\tfrac34] \\
2s - 1 & s\in[\tfrac34, 1].
\end{cases}
\]
Also folgt $(\gamma_1*\gamma_2)*\gamma_3\simeq\gamma_1*(\gamma_2*\gamma_3)$ aus \autoref{lem:reparam}.
\end{proof}

\begin{lemma}\label{lem:path-neutral}
Sei $X$ ein topologischer Raum und $x\in X$. Sei weiter $\varepsilon_x\colon I\to X$ der konstante Weg $\varepsilon(s) = x$. Für einen weiteren Weg $\gamma\colon[0,1]\to X$ mit $\gamma_1(1) = x$ ist dann $[\gamma*\varepsilon_x] = [\gamma]$. Ist hingegen $\gamma_1(0) = x$, so gilt $[\varepsilon_x*\gamma] = [\gamma]$.
\end{lemma}
\begin{proof}
Wir haben $(\gamma*\varepsilon_x)\circ\varphi = \gamma$ für
\[
\varphi\colon I\to I,\quad s\mapsto\begin{cases}
2s & s\in[0,\tfrac12] \\
1 & s\in[\tfrac12,1].
\end{cases}
\]
Also ist $\gamma*\varepsilon_x\simeq\gamma$ nach \autoref{lem:reparam}. Der zweite Teil folgt analog.
\end{proof}

\begin{lemma}\label{lem:path-inverse}
Sei wieder $\varepsilon_x$ der konstante Weg bei $x\in X$. Für einen Weg $\gamma\colon [0,1]\to X$ sei $\gamma'\colon I\to X$ definiert durch $\gamma'(s) = \gamma(1-s)$. Dann ist $[\gamma*\gamma'] = [\varepsilon_{\gamma(0)}]$ und $[\gamma'*\gamma] = [\varepsilon_{\gamma(1)}]$.
\end{lemma}
\begin{proof}
Wir haben eine Homotopie
\[
H\colon I\times I\to X,\quad (s,t)\mapsto\begin{cases}
\gamma(2s(1-t)) & s\in[0,\tfrac12] \\
\gamma(2(1-s)(1-t)) & s\in[\tfrac12, 1]
\end{cases}
\]
von $\gamma*\gamma'$ nach $\varepsilon_{\gamma(0)}$. Der zweite Teil folgt analog.
\end{proof}

\begin{definition}
Sei $X$ ein topologischer Raum. Eine \emph{Schleife bei $x\in X$} ist ein Weg $\gamma\colon I\to X$ mit $\gamma(0) = \gamma(1) = x$.
\end{definition}
\begin{definition}
Sei $X$ ein topologischer Raum mit $x_0\in X$. Die \emph{Fundamentalgruppe} $\pi_1(X,x_0)$ ist die Menge
\[
\pi_1(X,x_0) = \{[\gamma] : \textnormal{$\gamma\colon I\to X$ ist eine Schleife bei $x\in X$}\}.
\]
\end{definition}

\begin{theorem}
Die Menge $\pi_1(X,x_0)$ bildet mit der Verknüpfung $[\gamma_1]\cdot[\gamma_2] = [\gamma_1*\gamma_2]$ und dem neutralen Element $e = [\varepsilon_{x_0}]$ eine Gruppe.\proofomitted
\end{theorem}

Zum Beispiel ist die Fundamentalgruppe $\pi_1(\RR^n,0) = \{e\}$ trivial. In $\RR^n$ sind nämlich immer je zwei Wege mit den gleichen Start- und Endpunkten homotop. Insbesondere ist immer $\gamma\simeq \varepsilon_0$ für jede Schleife $\gamma$ bei $0\in\RR^n$. Wir werden später auch Räume kennenlernen, deren Fundamentalgruppe nichttrivial ist. Zum Beispiel zeigen wir später, dass $\pi_1(S^1,1)\isom\ZZ$.

Wir hatten in diesem Beispiel für $\RR^n$ den Basispunkt $0$ gewählt. Aber diese Wahl ist für die Fundamentalgruppe unerheblich. Sei nämlich $X$ ein topologischer Raum und $x_0,x_1\in X$ mit einem Weg $\tau\colon I\to X$ von $x_0$ nach $x_1$. Dann induziert $\tau$ eine Abbildung
\[
c(\tau)\colon \pi_1(X,x_1)\to\pi_1(X,x_0),\quad [\gamma]\mapsto [\tau*\gamma*\tau'].
\]
\begin{lemma}
Die Abbildung $c(\tau)\colon \pi_1(X,x_1)\to\pi_1(X,x_0)$ ist ein Gruppenisomorphismus.
\end{lemma}
\begin{proof}
Die Abbildung $c(\tau)$ ist ein Homomorphismus, denn
\[
c(\tau)([\gamma_1]\cdot[\gamma_2]) = [\tau * \gamma_1*\gamma_2*\tau'] = [\tau*\gamma_1*\tau'*\tau*\gamma_2*\tau'] = [\tau*\gamma_1*\tau']\cdot[\tau*\gamma_2*\tau'].
\]
Außerdem ist $c(\tau')$ die Umkehrabbildung von $c(\tau)$, denn
\[
(c(\tau)\circ c(\tau'))(\gamma) = [\tau * \tau' * \gamma * \tau * \tau'] = [\gamma]
\]
und
\[
(c(\tau')\circ c(\tau))(\gamma) = [\tau' * \tau * \gamma * \tau' * \tau] = [\gamma].\qedhere
\]
\end{proof}

Für wegzusammenhängende Räume $X$ hängt, bis auf Isomorphismus, die Fundamentalgruppe also nicht von der Wahl des Basispunkts ab. Wir schreiben in diesem Fall manchmal einfach $\pi_1(X)$.

\begin{definition}
Ein wegzusammenhängender topologischer Raum $X$ mit trivialer Fundamentalgruppe $\pi_1(X) = \{e\}$ heißt \emph{einfach zusammenhängend}.
\end{definition}

\end{document}
